\section{Examples}

\inputdir{green_function}

In order to understand and test the equation ~\ref{eq:gf} we set up  the experiments for two different distance $R$, the 
first one influenced by the near field ($R=80m$) and the second one dominated by the far field ($R=1000m$). For each of this
radius I change the angle $\theta$ shown in \rFg{setup} in intervals of $10\textdegree$ from $\theta=0\textdegree$ to 
$\theta=90\textdegree$, this implies that we have 10 cases for each $R$.

The function $X_o(t)$ used in this exercise is shown in ~\rfg{sin25Hz}. The medium has constant elastic properties which
are parametrized in terms of P and S wave velocity, and density. In this case $\alpha=2000m/s$, $\beta=1000m/s$ 
and $\rho=2500kg/m^3$.

Figures ~\ref{fig:GFtheta000-r0080} to ~\ref{fig:GFtheta090-r0080} shows the displacement waveforms for $R=80m$.
Figures ~\ref{fig:GFtheta000-r1000} to ~\ref{fig:GFtheta090-r1000} shows the displacement waveforms for $R=1000m$.

Then I show the hodograms in figures ~\ref{fig:hodo_p-s-GFtheta000-r0080} to ~\ref{fig:hodo_p-s-GFtheta090-r0080} and in figures
~\ref{fig:hodo_p-s-GFtheta000-r1000} to ~\ref{fig:hodo_p-s-GFtheta090-r1000}  for $R=80m$ and $R=1000m$ respectively. Each hodograms
is separated in two parts: P-wave arrival (red) and S-wave arrival time (blue). The P-wave arrival is in the interval $t \in [\frac{R}{\alpha},\frac{R}{\alpha}+T]$
. In the same way, the S-wave arrival is in the interval $t \in [\frac{R}{\beta},\frac{R}{\beta}+T]$, where $T$ is the period of the 
source function, in our case $T=0.04s$.




