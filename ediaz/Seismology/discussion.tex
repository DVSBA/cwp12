\section{Discussion}

In figure ~\ref{fig:GFtheta000-r0080} can be seen energy arriving at the $R/\beta$ arrival, eventhough the receiver vector is parallel
to the force $\vec{f}$. This anomalous arrival is coming from the near field contribution, and is polarized as a P-wave (parallel to the
direction of propagation). In figure ~\ref{fig:GFtheta090-r0080} can be seen the same anomalous arrival but at $R/\alpha$ arrival time, 
in this case this anomalous arrival is polarized as a S-wave. In this two cases the near field is also influenced by the far field waveforms.

In the far field we see a low energy arrival at $R/\beta$ travel time as shown in figure ~\ref{fig:GFtheta000-r1000}. This anomalous
arrival comes from the contribution of the near field, which is attenuated faster than the farfield terms. When the reciver is perpendicular
to the force, we also see a even lower energy arrival at $R/\alpha$ travel time, this can be seen in figure ~\ref{fig:GFtheta090-r1000}. 
The energy of the anomalous P-wave arrival in figure ~\ref{fig:GFtheta090-r1000} is half the energy of the S-wave arrival in figure
~\ref{fig:GFtheta000-r1000}. This amplitude difference is proportional to the ratio $\frac{\alpha}{\beta}$ which is $2$ in this case.


We can see in the hodograms for the near field, in figures ~\ref{fig:hodo_p-s-GFtheta000-r0080} to ~\ref{fig:hodo_p-s-GFtheta090-r0080}, 
that the displacement becomes highly non-linear when the angle $\theta$ is away from the $x_1$ and $x_3$ axes. The maximum of non-linear
 particle motion is in $\theta=45\textdegree$, this means that the minor axis of the formed semi-ellipse reaches its greatest magnitude.

Since the non-linearity of the particle motion is dominated by the near-field term. This means that this characteristic does not show up in great 
magnitude because the linear particle motion of the far field P and the far field S influence more the seismogram.

