\section{Discussion}

In figure ~\ref{fig:GFtheta000-r0080} can be seen energy arriving at the $R/\beta$ arrival, even though the receiver vector is parallel
to the force $\vec{f}$. This anomalous arrival is coming from the near field contribution, and is polarized as a P-wave (parallel to the
direction of propagation). In figure ~\ref{fig:GFtheta090-r0080} can be seen the same anomalous arrival but at $R/\alpha$ arrival time, 
in this case this anomalous arrival is polarized as a S-wave. In this two cases the near field is also influenced by the far field waveforms.

In the far field we see a low energy arrival at $R/\beta$ travel time as shown in figure ~\ref{fig:GFtheta000-r1000}. This anomalous
arrival comes from the contribution of the near field, which is attenuated faster than the far field terms. When the receiver is perpendicular
to the force, we also see a even lower energy arrival at $R/\alpha$ travel time, this can be seen in figure ~\ref{fig:GFtheta090-r1000}. 
The energy of the anomalous P-wave arrival in figure ~\ref{fig:GFtheta090-r1000} is half the energy of the S-wave arrival in figure
~\ref{fig:GFtheta000-r1000}. This amplitude difference is proportional to the ratio $\frac{\alpha}{\beta}$ which is $2$ in this case.


We can see in the hodograms for the near field, in figures ~\ref{fig:hodo_p-s-GFtheta000-r0080} to ~\ref{fig:hodo_p-s-GFtheta090-r0080}, 
that the displacement becomes highly non-linear when the angle $\theta$ is away from the $x_1$ and $x_3$ axes. The maximum of non-linear
 particle motion is in $\theta=45\textdegree$, this means that the minor axis of the formed semi-ellipse reaches its greatest magnitude.

Since the non-linearity of the particle motion is dominated by the near-field term. This means that this characteristic does not show up in great 
magnitude because the linear particle motion of the far field P and the far field S influence more the seismogram.


%Geometric setup
\inputdir{XFig}
 \plottwo{setup}{width=0.4\textwidth}{Geometric setup of the problem.}

\inputdir{green_function}
%X_o function
\plottwo{sin25Hz}{width=0.4\textwidth}{Time function used as the source $X_o(t)$.}


% Plots of displacement field:
\multiplot{10}{GFtheta000-r0080,GFtheta010-r0080,GFtheta020-r0080,GFtheta030-r0080,GFtheta040-r0080,GFtheta050-r0080,GFtheta060-r0080,GFtheta070-r0080,GFtheta080-r0080,GFtheta090-r0080}%
{height=0.2\textheight}{Seismograms for the near field (r=80m) for different angles $\theta$. Blue is $u_1$ component and red is the $u_3$ component of the displacement.}
\multiplot{10}{GFtheta000-r1000,GFtheta010-r1000,GFtheta020-r1000,GFtheta030-r1000,GFtheta040-r1000,GFtheta050-r1000,GFtheta060-r1000,GFtheta070-r1000,GFtheta080-r1000,GFtheta090-r1000}%
{height=0.2\textheight}{Seismograms for the far field (r=1000m) for different angles $\theta$. Blue is $u_1$ component and red is the $u_3$ component of the displacement.}


%Hodograms
\multiplot{10}{hodo_p-s-GFtheta000-r0080,hodo_p-s-GFtheta010-r0080,hodo_p-s-GFtheta020-r0080,hodo_p-s-GFtheta030-r0080,hodo_p-s-GFtheta040-r0080,hodo_p-s-GFtheta050-r0080,hodo_p-s-GFtheta060-r0080,hodo_p-s-GFtheta070-r0080,hodo_p-s-GFtheta080-r0080,hodo_p-s-GFtheta090-r0080}%
{height=0.2\textheight}{Hodograms for the near field (r=80m) for different angles $\theta$. In red is shown the part that correspond with the P-wave arrival travel time, in blue is shown the S-wave travel time.}
\multiplot{10}{hodo_p-s-GFtheta000-r1000,hodo_p-s-GFtheta010-r1000,hodo_p-s-GFtheta020-r1000,hodo_p-s-GFtheta030-r1000,hodo_p-s-GFtheta040-r1000,hodo_p-s-GFtheta050-r1000,hodo_p-s-GFtheta060-r1000,hodo_p-s-GFtheta070-r1000,hodo_p-s-GFtheta080-r1000,hodo_p-s-GFtheta090-r1000}%
{height=0.2\textheight}{Hodograms for the far field (r=1000m) for different angles $\theta$.In red is shown the part that correspond with the P-wave arrival travel time, in blue is shown the S-wave travel time.}

