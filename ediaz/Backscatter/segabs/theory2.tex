\section{Wave equation imaging conditions}
\inputdir{flat}



\multiplot{3}{cit05_ref1,citx05_ref1,cip05_ref1,img05_ref1}{height=0.18\textheight}{%
Synthetic model example: (a) time-lag gather at x=5km, (b) space-lag gather at x=5km, 
(c) common image point at x=5km, z=1.5km and (d) migrated image of one shot (in x=5km,z=0km) with receivers in the surface} 

\subsection{Conventional imaging condition}
The conventional imaging condition ~\citep{Claerbout:1985:IEI:3887}
is a zero time lag cross-correlation between the downgoing source wavefield and the upgoing 
receiver wavefield:
%
\beq
\R=\sum_{shots} \sum_{t} \US(\xx,t)\UR(\xx,t).
\label{eq:cic}
\eeq
%
 Equation~\ref{eq:cic} honors the single scattering or Born assumption. Under this assumption the transmitted
source wavefield generates secondary waves as it interacts with the medium discontinuities, which propagate in 
space and are recorded at the surface. This assumption means that both the source and receiver
wavefields carry only transmitted energy through interfaces between layers with different elastic properties.

A wavefield extrapolated with RTM could show, depending on the complexity of the geology, waves traveling in both
upward and downward directions, such as: diving waves, head waves and backscattered waves. The interaction between
 these type of waves contained in the source and receiver wavefields generates new events in the image which are 
commonly referred to as artifacts because they do not follow the geology (i.e earth reflectivity), which is the objective 
of the imaging process. The correlation between forward and backscattered waves is particularly strong when
sharp boundaries are present in the velocity model (e.g salt bodies).

If a sharp boundary is present in the model, we can decompose the source wavefield into transmitted 
 and the reflected energy originated at the sharp boundary:

\beq
\US(\xx,t)= \USr(\xx,t) +\USt(\xx,t),
\label{eq:ssplit}
\eeq
%
the superscripts ``t" and ``r" stands for transmitted and reflected energy respectively. 

The same idea can be applied to the receiver wavefield:

\beq
\UR(\xx,t)= \URr(\xx,t) +\URt(\xx,t).
\label{eq:rsplit}
\eeq
%
By taking advantage of the linearity of equation~\ref{eq:cic},  we
can split the conventional imaging condition as follows:

\beq
\R= \Rc{tt}+\Rc{tr}+ \Rc{rt} +\Rc{rr}.
\label{eq:cicsplit}
\eeq
%
Here, the first superscript is associated to the source wavefield and the second is associated to the
receiver wavefield, for example $\Rc{tr}$ is an image constructed with the transmitted source wavefield
and the reflected receiver wavefield.

By analyzing the individual contributions to the image, we can understand better how the backscattered events
are constructed in the image. This analysis is similar to the one of~\cite{fei:3130}, and~\cite{liu:S29}
whose objective is to filter out the non-geological portions of the image. Here, we approach the problem 
in a broader sense, by attempting to understand the physical meaning of the backscattered energy and its
uses for velocity model building.


\subsubsection{Backscattered events in the conventional image}

In order to gain an understanding of the RTM backscattered events we use a simple model with two-layers and strong velocity
contrast.~\rfg{img05_ref1} shows the common imaging condition for one shot at $x=5km$. We can see in this image strong 
backscattered energy, indicated with letter ``a", above the reflector located at $z=1.5km$.

To understand better the origin of the backscattered artifacts, we illustrate the wavefields used for imaging our simple model.
~\rfgs{wts-b05-150},~\ref{fig:wts-b05-275} and~\ref{fig:wts-b05-500} show three different snapshots
of the source wavefield.~\rfgs{wtr-b05-150},~\ref{fig:wtr-b05-275} and~\ref{fig:wtr-b05-500}
show the same snapshots for the receiver wavefield.~\rfgs{cross_corrs-b_r-b05-150},~\ref{fig:cross_corrs-b_r-b05-275} and
~\ref{fig:cross_corrs-b_r-b05-500} show the product between source and receiver wavefields for the same time snapshots. Finally,
~\rfgs{imgs-b_r-b05-150},~\ref{fig:imgs-b_r-b05-275} and~\ref{fig:imgs-b_r-b05-500} show the accumulated image (integration over time of 
the product between wavefields).

We can see in~\rfg{imgs-b_r-b05-150} the interaction between the transmitted source wavefield $\USt$, shown in~\rfg{wts-b05-150}, with 
the reflected receiver wavefield $\URr$, shown in~\rfg{wtr-b05-150}. In this case, the reflected receiver wavefield travels in perfect synchronization
with the transmitted source wavefield, therefore their product shown in~\rfg{cross_corrs-b_r-b05-150} stacks coherently in the imaging 
process generating the $\Rc{tr}$ contribution to the image $\R$. In the $\Rc{tr}$ image, the reflected 
receiver wavefield behaves as the transmitted source wavefield, which is the reason why the backscattered energy is imaged toward the source location. 
%
%
 In the partial image at $t=0.275s$ shown in~\rfg{imgs-b_r-b05-275}, we see how we start building the reflector image.
 The source wavefield shown in~\rfg{wts-b05-275} reflects, thus generating new backscattered events
corresponding to the  $\Rc{rt}$ image.
%
%
In the snapshot at $t=0.500s$, the reflector is imaged completely and 
 for remaining time we only add backscattered energy corresponding to the $\Rc{rt}$ image. In this case the 
reflected source wavefield behaves as the receiver wavefield and its energy maps toward the receivers. We can see after
 the imaging process is finished, shown in~\rfg{img05_ref1}, that the backscattered energy is maximum near the critical
 angle range (where the reflected source and receiver wavefields have maximum energy).

Using wavefield decomposition allow us to isolate the individual contributions of equation~\ref{eq:cicsplit}. 
~\rfg{img05_nn1} shows the cross-correlation between transmitted wavefields, an image which is due to the
earth reflectivity.~\rfgs{img05_nb1} and~\ref{fig:img05_bn1} show the images $\Rc{tr}$ and $\Rc{rt}$ corresponding
to the backscattered energy, and again we can observe how the backscattered energy
maps toward the source and the receivers, respectively. The image corresponding to the $\Rc{rr}$ case, shown 
in~\rfg{img05_bb1}, shows additional contribution to the reflectivity of the earth due to the cross-correlation
between reflected wavefields.~\cite{fei:3130} take  advantage of this analysis to define an image 
free from backscattered energy as $\R=\Rc{tt}+\Rc{rr}$. Here,
we want to understand better the meaning and uses of the other two partial images $\Rc{tr}$ and $\Rc{rt}$.

\multiplot{4}{wts-b05-150,wtr-b05-150,cross_corrs-b_r-b05-150,imgs-b_r-b05-150,wts-b05-275,wtr-b05-275,cross_corrs-b_r-b05-275,imgs-b_r-b05-275,wts-b05-500,wtr-b05-500,cross_corrs-b_r-b05-500,imgs-b_r-b05-500}{angle=90,width=0.16\textwidth}{%
Pictorial explanation of RTM imaging: rows 1, 2 and 3 correspond to three different snapshots at times $t_1=0.150s$, $t_2=0.275s$ and $t_3=0.500s$.
 Columns 1 to 4 correspond to the source wavefield, the receiver wavefield, the multiplication of the source and receiver wavefields,  and the accumulated image over time, respectively.}


%Image splitting:
\multiplot{2}{img05_nn1,img05_nb1,img05_bn1,img05_bb1}{height=0.12\textheight}{%
Illustration of the linearity of the conventional imaging condition. We can split the conventional image, figure~\ref{fig:img05_ref} in four %
separate images, $R^{tt}(\xx)$ (a), $R^{tr}(\xx)$ (b), $R^{tr}(\xx)$ (c), and $R^{rr}(\xx)$ (d), corresponding to the correlation of the %
transmitted and/or reflected components of the source and receiver wavefields}
\subsection{Extended imaging condition}

The extended imaging condition ~\citep{rickett:883,sava:S209,GPR:GPR888} is similar to the conventional IC except the cross-correlation
lags between source and receiver wavefield are preserved in the output:
\beq
\Re= \sum_{shots} \sum_{t} \US(\xx - \hh,t-\tau) \UR(\xx+\hh,t+\tau).
\label{eq:eic}
\eeq
Here $\hh$ represents the space lags, and $\tau$ represent the time-lag of the cross-correlation.
%
We can see that the conventional image is a special case of the extended image: $\R=R(\xx,{\bf 0},0)$.

By using extended images, we can measure the accuracy of the velocity model by analyzing the moveout of the events
 \citep{yang:S151}, and we can also perform transformations from the extended domain to angle domain 
\citep{sava:1065,sava:S209,sava:S131}. The extended images provide a measurement of the similarity between the source
 and receiver wavefields along space and time, so we can exploit these images to analyze and better understand the RTM backscattered events.

In equation~\ref{eq:eic} we observe an increase in the dimensionality of the image, from 3 dimensions to 7 dimensions
if we decide to extend the image in all directions. It is common to perform the analysis 
of extended images at limited locations in order to make this methodology feasible for large datasets. 
For cost considerations we often use one extension for common image gathers (CIG), for instance
 the time-lag axis ($\tau$) or in the space-lag axis ($\lambda_x$). We can also look into common image point gathers (CIP) where we 
fix an observation point ${\bf c}=(x,y,z)$ and analyze the image as a function of extensions $\hh,\tau$. 

~\rfgs{cit05_ref1},~\ref{fig:citx05_ref1} and~\ref{fig:cip05_ref1} we show a time-lag gather, a space-lag gather
 and a common image point respectively which represent subsets at fixed surface positions (for CIGs) or fixed space positions 
(for CIPs). Despite our model has only one reflector we can identify several events in the conventional and extended images. 
Letter ``a" indicates backscattered events, letter ``b" indicates the events produced by the cross-correlation of 
reflected wavefields, and finally letter ``c" indicates the cross-correlation between transmitted wavefields.

In presence of sharp velocity interfaces we can follow the same idea of equation~\ref{eq:cicsplit}, and construct 4 partial 
extended images:
\beq
\begin{split}
 \Re=  \\
\Rce{tt}+\Rce{tr}+\Rce{rt}+\Rce{rr}.
\label{eq:eicsplit}
\end{split}
\eeq
%
Again, this approach can help us to better understand the mapping patterns of backscattered events in extended images.
%Time-lag splitting":
\multiplot{4}{cit05_nn1,cit05_bn1,cit05_nb1,cit05_bb1}{height=0.2\textheight}{%
Illustration of the linearity of the time-lag extended imaging condition. We can split a time-lag gather, figure~\ref{fig:cit05_ref1} in four separate images
, $R^{tt}(z,\tau)$ (a), $R^{rt}(z,\tau)$ (b), $R^{tr}(z,\tau)$ (c) and $R^{rr}(z,\tau)$ (d), corresponding to the correlation of the transmitted %
and/or reflected components of the source and receiver wavefields}





\subsubsection{Time-lag common image gathers}

Using equation~\ref{eq:eicsplit}, we can analyze the individual contributions
for the time-lag gather shown in~\rfg{cit05_ref1}.~\rfg{cit05_nn1} shows the image $\Rct{tt}$,
in which we can observe a change in the slope of the events given by the abrupt velocity variation of the model. 
%
Above the reflector depth, the slope is controlled by the medium velocity of layer 1, whereas below the interface the slope is controlled
 by the velocity of the layer 2.~\rfgs{cit05_nb1} and \ref{fig:cit05_bn1} show the backscattered
 event contributions $\Rct{tr}$ and $\Rct{rt}$ respectively, which indicate that the backscattered event maps towards $\tau$=$0$
in the extended image. This means that we only get a contribution when we do not dislocate the wavefields by shifting them in time, 
thus reinforcing the idea of wavefield synchronization. We see in the time-lag gathers that the slope of the primaries is very different
from the backscattered slope. ~\cite{kaelin:3125} use the slope difference to filter the backscattered 
events in this domain, and to extract the conventional image from the filtered extended image $\R$=$R(\xx,\tau$=$0)$. 
%
%
\rfg{cit05_bb1} shows the $\Rct{rr}$ image, in this case the source wavefield is going in upward
direction and the receiver wavefield is going in downward direction, which is as if we change the order of cross-correlation in
equation~\ref{eq:eic}. This is why this events map in the time-lag gathers with a slope opposite to the primary above the
interface. The reflected waves only travel in the upper layer, which is why we observe this image above the reflector. In the $\Rct{rr}$
image we see two events that map with similar slope, one (the one below) has the exact opposite slope as the one shown by the primary reflection,
the other with slightly higher slope (therefore indicating faster velocity) corresponds to the interaction between head-waves produced by 
the medium velocity discontinuity and the reflected wavefields.

\subsubsection{Space-lag common image gathers}

%Space-lag splitting":
\multiplot{4}{citx05_nn1,citx05_bn1,citx05_nb1,citx05_bb1}{height=0.2\textheight}{%
Illustration of the linearity of the space-lag extended imaging condition. We can divide figure~\ref{fig:citx05_ref} in four images, %
 $R^{tt}(z,\lambda_x)$ (a), $R^{rt}(z,\lambda_x)$ (b), $R^{tr}(z,\lambda_x)$ (c) and $R^{rr}(z,\lambda_x)$ (d), corresponding to the correlation of %
the transmitted and/or reflected components of the source and receiver wavefields}

\rfg{citx05_ref1} shows a space-lag gather for the various combinations of the source and receiver wavefield components. 
We note that with the correct velocity model both primaries and backscattered events map to $\lambda_x=0$ since the velocity used for 
imaging is correct. 
%
\rfg{citx05_nn1} shows the $\Rcl{tt}$ image, where we see the energy correctly focused at $\lambda_x=0$. 
\rfgs{citx05_bn1} and~\ref{fig:citx05_nb1} show the backscattered events $\Rcl{rt}$ and $\Rcl{tr}$ in the space-lag gathers, 
which also map toward $\lambda_x=0$.
%
\rfg{citx05_bb1} shows the image coming from the reflected wavefields $\Rcl{rr}$, in this case the events are
visible only above the reflector because the waves travel only in the first layer.

\subsubsection{Common-image point gathers}
%Cip splitting":
\multiplot{2}{cip05_nn1,cip05_bn1,cip05_nb1,cip05_bb1}{height=0.2\textheight}{ %
Illustration of the linearity of the extended imaging condition for a common image point. We can decompose a CIP, figure~\ref{fig:cip05_ref}, in four images %
$R^{tt}(\hh,\tau)$ (a), $R^{rt}(\hh,\tau)$ (b), $R^{tr}(\hh,\tau)$ (c), $R^{rr}(\hh,\tau)$ (d) corresponding to the correlation
between transmitted and/or reflected components of the source and receiver wavefields}

The events involving backscattered energy are also visible in common image point gathers (CIPs).
%
\rfg{cip05_ref1} shows a CIP extracted at ${\bf c}$=$(5,1.5)km$.
\rfg{cip05_nn1} shows the CIP for the transmitted wavefields $\Rce{tt}$. We see the energy focusing at zero lag 
for the $\tau-\lambda_x$ panel. The $\lambda_z-\tau$ shows a kink produced by the abrupt change in velocity for the primaries, the primaries
are mapped for negative $\tau$.~\rfg{cip05_nb1}
 shows the $\Rce{tr}$ image, we can see a change in the $\lambda_z-\tau$ plane, where the backscattered is mapped to positives lags. Figure
~\ref{fig:cip05_bn1} shows the complementary backscattered energy, we see that the energy is mapped to negative $\lambda_z$ and positive $\tau$ lags.
The cip from the reflected wavefields shows weak energy with the same pattern of~\rfg{cip05_nn1}.




\section{Sensitivity to velocity errors}

\multiplot{3}{cit01_ref,cit02_ref,cit03_ref,cit04_ref,cit05_ref,cit06_ref,cit07_ref,cit08_ref,cit09_ref}{width=0.3\textwidth}{%
Model error sensitivity with time-lag gathers: (a) -12\%, (b) -9\%, (c) -6\%, (d) -3\%, (e) 0\%, (f) 3\%, %
(g) 6\%, (h) 9\% and (i) 12\% velocity perturbation in the top layer. The maximum energy of the backscattered events occur with correct 
velocity shown in panel (e).} 

\multiplot{3}{citx01_ref,citx02_ref,citx03_ref,citx04_ref,citx05_ref,citx06_ref,citx07_ref,citx08_ref,citx09_ref}{width=0.3\textwidth}{%
Model error sensitivity with space-lag gathers: (a) -12\%, (b) -9\%, (c) -6\%, (d) -3\%, (e) 0\%, (f) 3\%, (g) 6\%, (h) 9\% and (i) 12\% %
velocity perturbation in the top layer. Note that the maximum of backscattered energy happens with the correct velocity shown in panel (e).} 

\multiplot{3}{cip01_ref,cip02_ref,cip03_ref,cip04_ref,cip05_ref,cip06_ref,cip07_ref,cip08_ref,cip09_ref}{width=0.3\textwidth}{%
Model error sensitivity with time-lag gathers: (a) -12\%, (b) -9\%, (c) -6\%, (d) -3\%, (e) 0\%, (f) 3\%, %
(g) 6\%, (h) 9\% and (i) 12\% velocity perturbation in the top layer. The backscattered and primary events move away from zero lags,%
 with correct velocity, panel (e), all the events go through zero lags.} 



In this section we test the dependency of the backscattered energy with velocity errors using extended images. In previous sections 
we explained the wavefield synchronization idea for correct velocity, this implies that for correct velocity the backscattered energy maps 
toward zero lags. Now, we analyze the behaviour of backscattered events when we have velocity errors.
We test the sensitivity of the backscattered events with the same synthetic data discussed early, in this case we construct the images
with different models characterized by a constant error varying from -12\% to 12\% in the velocity of layer 1. We move the interface 
consistently with the velocity used for imaging, i.e we assume that the interface producing backscattered energy is placed in the model
according to the velocity in layer 1.
\rfgs{cit01_ref} to \ref{fig:cit09_ref} show time-lag gathers as a function of the velocity error.
 We can observe that the backscattered energy is still mapped vertically, but
away from $\tau$=$0$. The backscattered events in the time-lag gathers show a kinematic error, i.e these events move from positive $\tau$
for negative errors to negative $\tau$ values for positive errors. 
%
\rfgs{citx02_ref} to~\ref{fig:citx09_ref} show a similar display for space-lag gathers. 
In this case we can see how both backscattered and primaries energy maps away from $\lambda_x$=$0$ when we introduce an error in the model. 
 In space-lag gathers the backscattered energy maps symetrically away from zero lag with incorrect velocties. 
%
Finally,~\rfgs{cip02_ref} to~\ref{fig:cip09_ref} show the sensitivity of extended CIPs to velocity errors. We see that for incorrect velocity, the events
move away zero-lags. In the CIPs, even with incorrect velocity the primary reflections go trough zero-lag, the events in the $\tau-\lambda_x$ plane shows a
moveout (i.e the energy is not mapped symmetrically respect zero lag). 
The velocity errors split the backscattered energy in the $\lambda_z-\tau$ plane, some of the energy going trough zero space-lag and other part
of the energy does not go through zero lag.

We could use the information contained in the extended images to design objective functions (OF) that exploits the presence of backscattered events. 
Minimizing this OF, e.g by wavefield tomography, optimizes the sharp interface positioning (e.g top of salt) and the sediments velocity above the interface.
A straightforward approach based on differential semblance optimization~\citep{shen:2132} can be adapted to use
 the backscattered energy seen away from zero lags by defining the objective functions for time-lag gathers
\beq
 J_{\tau}= \frac{1}{2} \lnorm{P_{\tau}\left [R^{tr}(\xx,\tau)+R^{rt}(\xx,\tau)\right ]}^2_2,
\label{eq:of1}
\eeq 
and for the space-lag gathers,
\beq
 J_{\lambda_x}= \frac{1}{2} \lnorm{P_{\tau}\left [R^{tr}(\xx,\lambda_x)+R^{rt}(\xx,\lambda)\right ]}^2_2.
\label{eq:of2}
\eeq 
%
Here $P_{\tau}=|\tau|$ and $P_{\lambda_x}=|\lambda_x|$ are functions that penalize the backscattered energy away 
from zero lags, thus defining a residual we need to minimize trough inversion.

For common image points we can use:
\beq
J_{\bf c} =  \frac{1}{2} \lnorm{P(\hh,\tau)\left [R^{tr}(\hh,\tau)+R^{rt}(\hh,\tau)\right ]}^2_2.
\eeq
%
Here, $P(\hh,\tau)$ is the penalty function for CIPs.

The penalty function is designed to measure the deviation or error between actual extended images and our notion for correct 
extended images. For CIGs we have a definite criterion, we know that the backscattered energy has to map
to zero lag, that is why we can use the absolute value as penalty function. However, for CIPs the 
penalty operator is more complex. We use the correct CIP as reference for constructing the penalty function $P(\hh,\tau)$ similar
to the one proposed by~\citep{tony:cwp12}. The correct CIP, shown in~\rfg{cip05_ref} has the right focusing within the acquisition limitations.

The objective function for our problem are shown in~\rfgs{OF_cit},~\ref{fig:OF_cix} and~\ref{fig:OF_cip} for time-lag gathers, space-lag gathers and
common image point gathers respectively. One can see that in all three cases the OF minimizes at the correct model. If we want to optimize the model such as 
we maximize the backscattered events we need to consider two variables: the velocity model and the interface geometry. The strength of the backscattered
events depends on two variables: the velocity model and sharp interface positioning. In our example the sharp interface depends linearly with the 
velocity model.


\multiplotcol{1}{OF_cit,OF_cix,OF_cip}{height=0.2\textheight}{Objective functions for time-lag CIG $ J_{\tau}$ (a), space-lag CIG $ J_{\lambda_x}$ (b) and
CIP gathers $J_{\bf c})$.}

%The objective functions for CIGs are shown in~\rfgs{OF_cit} and~\ref{fig:OF_cix} for time-lag gathers
%and space-lag gathers respectively. One can see that the function minimize at the correct velocity model. In the definition 
%of the OF we only use the backscattered energy $\Rct{tr}+\Rct{rt}$ and $\Rcl{tr}+\Rcl{rt}$ for time-lag and space. We separate the wavefield contribution using
%wavefield up and down decomposition, alternatively we could use slope filtering in the extended images. This is a 
%robust and cost effective operation since the various events in the gathers are characterized
%by distinct slopes as shown by~\cite{kaelin:3125}.
