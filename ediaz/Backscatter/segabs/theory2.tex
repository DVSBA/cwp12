\section{Wave equation imaging conditions}
\inputdir{flat}



\multiplot{3}{cit05_ref1,citx05_ref1,cip05_ref1,img05_ref1}{height=0.11\textheight}{%
Synthetic model example: (a) time-lag gather at x=5km, (b) space-lag gather at x=5km, 
(c) common image point at x=5km, z=1.5km and (d) migrated image of one shot} 

\subsection{Conventional imaging condition}
The conventional imaging condition ~\citep{Claerbout:1985:IEI:3887}
is a zero time lag cross-correlation between the source wavefield and the 
receiver wavefields:
%
\beq
\R=\sum_{shots} \sum_{t} \US(\xx,t)\UR(\xx,t).
\label{eq:cic}
\eeq

A wavefield extrapolated with RTM could show, depending on the complexity of the geology, waves traveling in both
upward and downward directions, such as diving waves, head waves and backscattered waves.~
The correlation between forward and backscattered waves is particularly strong when
sharp boundaries are present in the velocity model (e.g. salt bodies).

If a sharp boundary is present in the model, we can decompose the source wavefield into transmitted 
 and reflected energy that originates at the sharp boundary:

\beq
\US(\xx,t)= \USr(\xx,t) +\USt(\xx,t),
\label{eq:ssplit}
\eeq
%
where the superscripts t and r stand for transmitted and reflected energy, respectively. 

The same idea can be applied to the receiver wavefield:

\beq
\UR(\xx,t)= \URr(\xx,t) +\URt(\xx,t).
\label{eq:rsplit}
\eeq
%
By taking advantage of the linearity of equation~\ref{eq:cic},  we
can split the conventional imaging condition as follows:

\beq
 \R = \Rc{tt} +\Rc{rr} + \Rc{tr}+ \Rc{rt}.
\label{eq:cicsplit}
\eeq
%
Here, the first superscript is associated with the source wavefield and the second is associated to the
receiver wavefield. For example, $\Rc{tr}$ is an image constructed with the transmitted source wavefield
and the reflected receiver wavefield.

By analyzing the individual contributions to the image, we can better understand how the backscattered events
are constructed in the image. This analysis is similar to the one of~\cite{fei:3130} and~\cite{liu:S29}
whose objective is to filter out the non-geological portions of the image. Here, we approach the problem 
in a broader sense by attempting to understand the physical meaning of the backscattered energy and its
uses for velocity model building.


In order to gain an understanding of the RTM backscattered events, we use a simple model with two-layers and strong velocity
contrast.~\rfg{img05_ref1} shows the image obtained with the conventional imaging condition for one shot at $x=5km$. This image has strong 
backscattered energy, indicated with letter ``a", above the reflector located at $z=1.5km$.

\subsection{Extended imaging condition}

The extended imaging condition~\citep{rickett:883,sava:S209,GPR:GPR888} is similar to the conventional imaging condition
 except the cross-correlation lags between source and receiver wavefield are preserved in the output:
\beq
\Re= \sum_{shots} \sum_{t} \US(\xx - \hh,t-\tau) \UR(\xx+\hh,t+\tau).
\label{eq:eic}
\eeq
Here $\hh$ and $\tau$ represent the space-lags and time-lags, respectively, of the cross-correlation.
%
The conventional image is a special case of the extended image $\R=R(\xx,{\bf 0},0)$.

By using extended images, we can measure the accuracy of the velocity model by analyzing the moveout of the events
 \citep{yang:S151}, and we can perform transformations from the extended to the angle domain 
\citep{sava:1065,sava:S209,sava:S131}. The extended images provide a measurement of the similarity between the source
 and receiver wavefields along space and time, so we can exploit these images to analyze the RTM backscattered events.


~\rfgs{cit05_ref1},~\ref{fig:citx05_ref1} and~\ref{fig:cip05_ref1} show a time-lag gather, a space-lag gather,
 and a common image point, respectively, which represent subsets at fixed surface positions (for CIGs) or fixed space positions 
(for CIPs). Despite the fact that our model has only one reflector, we can identify several events in the conventional and extended images. 
Letter ``a" indicates backscattered events, letter ``b" indicates the events produced by the cross-correlation of 
reflected wavefields, and letter ``c" indicates the cross-correlation between transmitted wavefields.

In the presence of sharp velocity interfaces we can use the concept of equation~\ref{eq:cicsplit}, and construct four partial 
extended images:

\beq
 \Re= \Rce{tt}+\Rce{rr}+\Rce{tr}+\Rce{rt}.
\label{eq:eicsplit}
\eeq

%Time-lag splitting":
\multiplot{2}{cit05_nn1,cit05_bn1,cit05_nb1,cit05_bb1}{height=0.12\textheight}{%
Illustration of the linearity of the time-lag extended imaging condition. We can split a time-lag gather, Figure~\ref{fig:cit05_ref1}, in four separate images,
 $R^{tt}(z,\tau)$ (a), $R^{rt}(z,\tau)$ (b), $R^{tr}(z,\tau)$ (c) and $R^{rr}(z,\tau)$ (d), corresponding to the correlation of the transmitted %
and/or reflected components of the source and receiver wavefields}



\section{Sensitivity to velocity errors}

\multiplot{3}{cit00_ref,cit10_ref,citx00_ref,citx10_ref,cip00_ref,cip10_ref}{height=0.12\textheight}{%
Sample gathers with velocity errors: (a) and (b) time-lag gathers with -12\% and +12\% velocity error, (c) and (d) time-lag gathers with -12\% and +12\% velocity error,
(e) and (f) CIP gathers with -12\% and +12\% velocity error } 

In this section, we test the dependency of the backscattered energy on velocity errors using extended images. In the previous sections 
we have explained the concept of wavefield synchronization for correct velocity, which implies that for correct velocity the backscattered energy maps 
toward zero lags. Here, we analyze the behavior of backscattered events in the presence of velocity errors.
We test the sensitivity of the backscattered events with the same synthetic data discussed previously. In this case, we construct the images
with different models characterized by a constant error varying from -12\% to +12\% in layer 1. We keep the interface 
consistent with the velocity used for imaging, i.e. we assume that the interface producing backscattered energy is placed in the model
according to the velocity in layer 1.

We could use the information contained in the extended images to design objective functions (OF) that exploit the presence of backscattered events. 
Minimizing such OF, e.g. by wavefield tomography, optimizes the sharp interface positioning (e.g. the top of salt) and the sediments velocity above it.
A straightforward approach based on differential semblance optimization~\citep{shen:2132} can be adapted to use
 the backscattered energy seen away from zero lags by defining the objective functions for time-lag gathers
\beq
 J_{\tau}= \frac{1}{2} \lnorm{P(\tau)\left [R^{tr}(\xx,\tau)+R^{rt}(\xx,\tau)\right ]}^2_2,
\label{eq:of1}
\eeq 
and for the space-lag gathers,
\beq
 J_{\lambda_x}= \frac{1}{2} \lnorm{P(\lambda_x)\left [R^{tr}(\xx,\lambda_x)+R^{rt}(\xx,\lambda)\right ]}^2_2.
\label{eq:of2}
\eeq 
%
Here $P(\tau)=|\tau|$ and $P(\lambda_x)=|\lambda_x|$ are functions that penalize the backscattered energy away 
from zero lags, thus defining the residual that we need to minimize trough inversion.

For common image points we can use the objective function
\beq
J_{\bf c} =  \frac{1}{2} \lnorm{P(\hh,\tau)\left [R^{tr}(\hh,\tau)+R^{rt}(\hh,\tau)\right ]}^2_2.
\eeq
%
Here, $P(\hh,\tau)$ is the penalty function for CIPs.

The penalty function is designed to measure the deviation ojoinrror between actual extended images and our notion for correct 
extended images. For CIGs we have a definite criterion, we know that the backscattered energy has to map
to zero lag, that is why we can use the absolute value as penalty function. However, for CIPs the 
penalty operator is more complex. We use the correct CIP as reference for constructing the penalty function $P(\hh,\tau)$ similar
to the one proposed by~\citep{tony:cwp12}. The correct CIP, shown in~\rfg{cip05_ref1} has the right focusing within the acquisition limitations.
More generally, we could use a demigration/migration process to assess correct focusing at a given CIP position, and to infer the shape of 
the penalty operator.

The objective functions for our synthetic example are shown in~\rfgs{OF_join} for time-lag gathers, space-lag gathers and
common image point gathers respectively. One can see that in all three cases the OF minimizes at the correct model. If we want to optimize the model such as 
we maximize the backscattered events we need to consider two variables: the velocity model and the interface geometry. 
 In our example the sharp interface depends linearly with the velocity model.


\multiplot{1}{OF_join}{width=0.4\textwidth}{Objective functions for time-lag gathers $ J_{\tau}$ (a), space-lag gathers $ J_{\lambda_x}$ (b) and
CIP gathers $J_{\bf c}$.}

