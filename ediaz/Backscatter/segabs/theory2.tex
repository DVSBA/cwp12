\inputdir{flat}
\multiplot{2}{cit05_ref1,citx05_ref1}{height=0.14\textheight}{%
Synthetic example: (a) time-lag and (b) space-lag gathers at x=5km.} 
\multiplot*{2}{cip05_ref1,img05_ref1}{height=0.14\textheight}{%
Synthetic  example: (a) common image point at $(x,z)=(5.0,1.5)$km and (b) migrated image of one shot.} 
\section{Theory}
The conventional imaging condition (CIC) ~\citep{Claerbout:1985:IEI:3887}
is a zero time-lag cross-correlation between the source wavefield and the 
receiver wavefields:
%
\beq
\R=\sum_{shots} \sum_{t} \US(\xx,t)\UR(\xx,t).
\label{eq:cic}
\eeq

A wavefield extrapolated with RTM could show, depending on the complexity of the geology, waves traveling in both
upward and downward directions, such as diving waves, head waves and backscattered waves.~
The correlation between forward and backscattered waves is particularly strong when
sharp boundaries are present in the velocity model (e.g. for salt bodies).

If a sharp boundary is present in the model, we can decompose the source wavefield into transmitted 
 and reflected energy that originates at the sharp boundary:
%
\beq
\US(\xx,t)= \USr(\xx,t) +\USt(\xx,t),
\label{eq:ssplit}
\eeq
%
where the superscripts t and r stand for transmitted and reflected energy, respectively. 
The same idea can be applied to the receiver wavefield:
%
\beq
\UR(\xx,t)= \URr(\xx,t) +\URt(\xx,t).
\label{eq:rsplit}
\eeq
%
By taking advantage of the linearity of equation~\ref{eq:cic},  we
can split the conventional imaging condition as follows:
%
\beq
 \R = \Rc{tt} +\Rc{rr} + \Rc{tr}+ \Rc{rt}.
\label{eq:cicsplit}
\eeq
%
Here, the first superscript is associated with the source wavefield and the second is associated with the
receiver wavefield. For example, $\Rc{tr}$ is an image constructed with the transmitted source wavefield
and the reflected receiver wavefield.

This analysis can be used as well with the extended imaging condition (EIC)
~\citep{rickett:883,sava:S209,GPR:GPR888}. The EIC is similar to the CIC except that the cross-correlation
lags between source and receiver wavefields are preserved in the output as follows:
\beq
\Re= \sum_{shots} \sum_{t} \US(\xx - \hh,t-\tau) \UR(\xx+\hh,t+\tau).
\label{eq:eic}
\eeq
Here $\hh$ and $\tau$ represent the cross-correlation space-lags and time-lags, respectively.
%
The conventional image is a special case of the extended image $\R=R(\xx,{\bf 0},0)$.

By using extended images, we can measure the accuracy of the velocity model by analyzing the moveout of the events
 \citep{yang:S151}, and we can perform transformations from the extended to the angle domain 
\citep{sava:1065,sava:S209,sava:S131}. The extended images provide a measurement of the similarity between the source
 and receiver wavefields along space and time, so we can exploit these images to analyze the RTM backscattered events.

In the presence of sharp boundaries, we can also construct four partial extended images:
%
\begin{align}
\nonumber \Re &= \Rce{tt}+\Rce{rr} \\
     &+\Rce{tr}+\Rce{rt}.
\label{eq:eicsplit}
\end{align}
%
By analyzing the individual contributions to the image and extended image, we can better understand the behavior of the backscattered events.
 This analysis is similar to the one of~\cite{fei:3130} and~\cite{liu:S29}
whose objective is to filter out the non-geological portions of the image. Here, we approach the problem 
in a broader sense by attempting to understand the physical meaning of the backscattered energy and its
uses for velocity analysis.


In order to gain an understanding of the RTM backscattered events, we use a simple model with two-layers and a strong velocity
contrast.~\rfg{img05_ref1} shows the image obtained with the conventional imaging condition for one shot at $x=5km$. This image has strong 
backscattered energy (indicated with letter ``a") above the reflector located at $z=1.5km$.~\rfgs{cit05_ref1},~\ref{fig:citx05_ref1} and~\ref{fig:cip05_ref1}
show a time-lag gather, a space-lag gather and a common image (CIP) point gather. The backscattered energy $\Rce{tr}+\Rce{rt}$ (denoted with letter ``a")
maps toward zero lag for time-lag and space-lag gathers, as shown in~\rfgs{cit05_ref1} and~\ref{fig:citx05_ref1}, respectively. 
%
This mapping to zero lag means that the reflected wavefields map in perfect synchronization with the transmitted wavefields;
therefore they cannot be dissociated in the imaging condition. This synchronization is achieved only because we use the correct velocity model to 
obtain these images.
%
We can also identify the cross-correlation between the reflected wavefields (denoted by letter ``b") in~\rfgs{cit05_ref1} and
\ref{fig:citx05_ref1} for time-lag and space-lag gathers, respectively. 
In the time-lag gathers, this event has an opposite slope compared with the one of the primaries, 
as if we changed the cross-correlation order in equation~\ref{eq:eic}. 


The backscattered events (identified with letter ``a") also appear in CIP gathers, as shown in~\rfg{cip05_ref1}. Both
backscattered contributions map to $\tau>0$ in the $\tau-\lambda_z$ plane. However, they map as two different events, whereas for time-lag and space-lag 
gathers both map to zero lag. In the $\Rce{tr}$ image, they map to $\tau>0$ and $\lambda_z>0$, and in the $\Rce{rt}$ image they map to $\tau>0$ and $\lambda_z<0$.

%For isolating the events in the output image we can use a wavefield decomposition approach to split the wavefields as
%shown in equations~\ref{eq:ssplit} and~\ref{eq:rsplit}, once the wavefields are split we can obtain the individual images
%shown in equation~\ref{eq:eicsplit}. We show an example of this decomposition in~\rfgs{cit05_nn} to~\ref{fig:cit05_bb}.

%Time-lag splitting":
%\multiplot*{4}{cit05_nn1,cit05_bn1,cit05_nb1,cit05_bb1}{height=0.12\textheight}{%
%Illustration of the linearity of the time-lag extended imaging condition. We can split a time-lag gather, Figure~\ref{fig:cit05_ref1}, in four separate images,
% $R^{tt}(z,\tau)$ (a), $R^{rt}(z,\tau)$ (b), $R^{tr}(z,\tau)$ (c) and $R^{rr}(z,\tau)$ (d), corresponding to the correlation of thetransmitted %
%and/or reflected components of the source and receiver wavefields}



\section{Sensitivity to velocity errors}

\multiplot{2}{cit00_ref,cit10_ref,citx00_ref,citx10_ref,cip00_ref,cip10_ref}{height=0.14\textheight}{%
Sample gathers with -12\% and +12\% velocity error: (a)-(b) time-lag gathers, (c)-(d) space-lag gathers, (e)-(f) CIP gathers.} 

In this section, we test the dependency of the backscattered energy on velocity errors using extended images.
We analyze the behavior of backscattered events in the presence of velocity errors.
We test the sensitivity of the backscattered events with the same synthetic data discussed previously. In this case, we construct the images
with different models characterized by a constant error varying from -12\% to +12\% in layer 1. We keep the interface 
consistent with the velocity used for imaging, i.e. the interface producing backscattered energy is placed in the model
according to the velocity in layer 1.

Figures~\ref{fig:cit00_ref} and~\ref{fig:cit10_ref} show time-lag gathers for a -12\% and +12\% velocity error, respectively. One can see
that the backscattered energy does not map to $\tau=0$ because the wavefields are no longer synchronized. For negative errors the artifacts map to $\tau>0$, whereas
for positive errors they map to $\tau<0$. Figures~\ref{fig:citx00_ref} and~\ref{fig:citx10_ref} 
show space-lag gathers for the same velocity errors. In this case we see that the backscattered energy maps symmetrically away with respect to $\lambda_x=0$.
Figures~\ref{fig:cip00_ref} and~\ref{fig:cip10_ref} show CIP gathers for the same velocity errors.
One can see that the velocity errors split the backscattered energy in the $\lambda_z-\tau$ and $\lambda_x-\lambda_z$ planes; 
some of the energy goes trough zero space-lag, while other part of the energy does not.


We can use the kinematic information from backscattered events contained in the extended images 
to design objective functions (OF) that exploit the presence of backscattered events. To isolate the backscattered
events, we use wavefield decomposition to obtain the individual contributions shown in equation~\ref{eq:eicsplit}. We use the images
$\Rce{tr}$ and $\Rce{rt}$ which contain backscattered energy.
Minimizing such OF, e.g. by wavefield tomography, optimizes the sharp interface positioning (e.g. the top of salt) and the sediment velocity above it.
A straightforward approach based on differential semblance optimization~\citep{shen:2132} can be adapted to use
 the backscattered energy seen away from zero lags by defining objective functions for time-lag gathers
\beq
 J_{\tau}= \frac{1}{2} \lnorm{P(\tau)\left [R^{tr}(z,\tau)+R^{rt}(z,\tau)\right ]}^2_2,
\label{eq:of1}
\eeq 
and for space-lag gathers,
\beq
 J_{\lambda_x}= \frac{1}{2} \lnorm{P(\lambda_x)\left [R^{tr}(z,\lambda_x)+R^{rt}(z,\lambda)\right ]}^2_2.
\label{eq:of2}
\eeq 
%
The functions $P(\tau)=|\tau|$ and $P(\lambda_x)=|\lambda_x|$ penalize the backscattered energy away 
from zero lags, thus defining the residual that we need to minimize trough inversion.

For common image points we can use
\beq
J_{\bf c} =  \frac{1}{2} \lnorm{P(\hh,\tau)\left [R^{tr}(\hh,\tau)+R^{rt}(\hh,\tau)\right ]}^2_2.
\eeq
%
The penalty functions are designed to measure the deviation error between actual and ideally focused 
extended images. For CIGs we have a definite criterion: we know that the backscattered energy has to map
to zero lag. However, for CIPs the penalty operator is more complex. We use a correct CIP as reference for constructing the penalty function $P(\hh,\tau)$.
 The correct CIP, shown in~\rfg{cip05_ref1}, has the correct focusing within the acquisition limitations.
We could use a demigration/migration process to assess correct focusing at a given CIP position, and to infer the shape of 
the penalty operator,  similar to the method proposed by~\cite{tony_seg:cwp12}
%More generally, we could use a demigration/migration process to assess correct focusing at a given CIP position, and to infer the shape of 
%the penalty operator.

The objective functions for our synthetic example are shown in~\rfg{OF_join} for time-lag CIGs, space-lag CIGs, and
common image point gathers, respectively. In all three cases the OF minimizes at the correct model. 

\plot{OF_join}{width=0.4\textwidth}{Normalized objective functions $J_{\tau}$ (red), $J_{\lambda_x}$ (blue) and
$J_{\bf c}$ (green).}

