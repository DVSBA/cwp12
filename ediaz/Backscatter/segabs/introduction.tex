\section{Introduction}

Almost 30 years ago  ~\cite{baysal:1514}, ~\cite{whitmore:382}
 and ~\cite{GPR:GPR413} introduced the concept of reverse time migration (RTM). 
It wasn't until the late 90's and mainly last decade that the computational advances
allowed the geophysical community to use this technology in practice. In general, and 
especially in complex geological settings,  RTM produces better images. This 
characteristic is due to its ability to image without dip limitations. The
reason for the RTM success in complex settings is the two-way nature of its 
engine, a non-factorized wave equation. Basically, a wavefield used in RTM
has the freedom to move everywhere, whereas other methods are limited to one
direction as in one-wave equation migration, or use only one specific arrival 
as in Kirchoff migration.

Another characteristic of RTM images is the presence of low 
wavenumber events or ``artifacts" that are not correlated with the geology. The two-way
wave equation allow us to cross-correlate energy traveling in both directions as diving waves,
head waves and backscattered waves, thus producing low wavenumber events. These
``artifacts" are specially more evidents in the presence of sharp velocity boundaries (e.g. salt 
models). 

In the past decade, several strategies to attenuate these events have been presented. 
We can classify them in two main families: pre imaging condition and post imaging condition. 

In the pre imaging condition family we can find wavefield decomposition approaches
~\citep[]{liu:S29,fei:3130}. The purpose of wavefield decomposition is to honor the original
assumptions of the imaging condition by cross-correlating downgoing energy with upgoing energy.
We can also find modeling approaches to attenuate reflections in sharp velocity boundaries 
~\citep{fletcher:2049}, or eliminate reflected energy in the modeled wavefields using 
impedance matching ~\citep{baysal:132}. This last method only works with zero-offset (stacked) data.  

The other family of noise attenuation techniques is applied after imaging, this is faster because
it is done after stacking over shots. A common approach is to apply a Laplacian filter
to the image ~\citep{youn:246}. A second strategy proposed by ~\cite{guitton:S19} is a signal-noise
separation by least squares, where the signal is defined as the reflectivity and the noise is the low 
wavenumber energy of the image. 

The low wavenumber events have a special way of mapping into extended images. ~\cite{kaelin:3125} propose
a filter that takes advantage of this feature. The filter is applied to the time-lag gathers, where the
backscattered energy maps into vertical events (discussed later), and therefore is easy to remove.

In this report we analyze the information carried by the backscattered energy. We examine as well 
how it maps into the extended images. We show that by analyzing this piece of information we 
can perform a quality control of sharp boundary placement in the model (e.g. sediment-salt interface). 
We use a very simple 2 layers model to explain how the backscattered events are built in the imaging
process. Then, we test these ideas with an example of the Sigsbee model.

% Talk about Reverse Time Migration
 
% Talk about different techniques to attenuate artifacts

% Introduce the paper objective


%~\cite{claerbout:467} 
%~\cite{baysal:1514}
%~\cite{whitmore:382} 
%~\cite{guitton:S19} 
%~\cite{kaelin:3125} 
%~\cite{fletcher:2049} 
%~\cite{sava:S209} 
%~\cite{fei:3130} 
%~\cite{GPR:GPR413} 
%~\cite{baysal:132}
