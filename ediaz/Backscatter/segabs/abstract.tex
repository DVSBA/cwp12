\begin{abstract}
Reverse time migration (RTM) backscattered events are produced by the cross-correlation between waves reflected from 
sharp interfaces (e.g. the top of salt bodies). 
%These events, along with head waves and diving waves, produce
%the so-called RTM artifacts, which are visible as low wavenumber energy on migrated images. 
Commonly, these events are seen as a drawback for the RTM method because they obstruct the image of the geologic
structure.
Many strategies have been developed to filter out the artifacts from the conventional image. However,
 these events contain information that can be used to analyze kinematic synchronization between source and receiver wavefields reconstructed 
in the subsurface. Numeric and theoretical analysis 
indicate the sensitivity of the backscattered energy to velocity accuracy: an accurate velocity model
maximizes the backscattered artifacts. The analysis of RTM extended images can be used as a quality control tool 
and as input to velocity analysis designed to constrain salt models and sediment velocity.
\end{abstract}
