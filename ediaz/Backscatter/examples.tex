\section{Examples}
In order to analyze the backscattered energy I use a simple two layers model. I then 
use two cases: constant density and hight  velocity constrast, constant velocity and density
contrast.  

For each backscatering case I analyze the behaivoour opf the backscatering energy for different
velocity errors (from -25\% to +25\% in intervals of 5\%). By considering the hard interface in 
the density model we have the ability to turn off and on the back scatered energy in the receiver
 and source wavefields.

To exemplify this workflow we can consider the following example:
\begin{enumerate}
\item Propagate source to model $V(\bf{x}), \rho(\bf{x})$ to obtain $\US=\US^b +\US^n$
\item Propagate source to model $V(\bf{x}), \rho_o(\bf{x})$, where
$\rho_o$ is constant, to obtain $\US^n$.
\item Obtain $\US^b= \US -\US^n$.
\end{enumerate}

The same workflow applies to the receiver wavefield. By doing this we have the ability to separate all 
the cross-correlation cases that contributes to an image $R(\bf{x})$.


\inputdir{flat2}

\multiplot{3}{vel05,den05,img05_ref}{width=0.4\textwidth}{example....}


\multiplot*{4}{img05_nn,img05_nb,img05_bn,img05_bb}{width=0.4\textwidth}{Example of
  multiplot.}
