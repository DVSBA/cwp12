\section{Examples}
In order to analyze the backscattered energy I use a simple two layers model. I study
 two cases: (1)  constant velocity and density contrast and,
  (2) constant density and velocity contrast.  

For the case of the hard interface in the density model, I study the contribution of each case of 
equation ~\ref{eq:cases}. For the hard interface in the velocity model I study the potential
to observe kinematic changes in the backscattered energy.

In order to analyze the contribution of each case of equation~\ref{eq:cases}, I use the following flow:
\begin{enumerate}
\item Propagate source to model $V(\bf{x}), \rho(\bf{x})$ to obtain $\US=\US^b +\US^n$
\item Propagate source to model $V(\bf{x}), \rho_o(\bf{x})$, where
$\rho_o$ is constant, to obtain $\US^n$.
\item Obtain $\US^b= \US -\US^n$.
\end{enumerate}

The same workflow applies to the receiver wavefield. By doing this we have the ability to separate all 
the cross-correlation cases that contributes to an image $R(\bf{x})$.


\inputdir{flat2}
\subsection{Understanding the backscattered energy.}

In order to obtain an understanding of the phenomena, I do a very simple 2D example. I consider
a constant velocity medium as shown in figure~\rfn{vel05} and, a two layered density model in figure \rfn{den05}. I use
the hard interface in the density model to split in an easy way the cases shown in equation \ref{eq:cases}. 
The experiment is composed by one shot with a receiver array in the complete surface. The conventional image
given by equation~\ref{eq:IC} is shown in figure~\rfn{img05_ref}. 

\multiplot{4}{vel05,den05,img05_ref,cit05_ref}{width=0.4\textwidth}{Constant velocity model of 2.2km/s (a), density model (b) 
and retreived image (c) and time-lag gather (d). The vertical line shows the location of the time lag gather.}

With this model with variable density we can produce all the cases shown in equation ~\ref{eq:cases}.  In figures ~\rfn{img05_nn} to ~\rfn{img05_bb},
 I show the individual contributions. We can see in~\rfn{img05_nb} how the backscattered energy is concentrated toward the source, this happens because
the backscattered receiver field acts as the source wavefield. Similar analysis can be applied in figure ~\ref{fig:img05_bn}, where
the source wavefield acts as the receiver wavefield. One can see that all the cases contributes at some degree with the geology, of course the most important
contribution is given by $R(\bf{x})^{nn}$.

We can also analyze how does the backscattering images look in the extended image domain. \rFg{cit05_ref} shows a time-lag
gather located at the center of the model (yellow vertical line in figure ~\ref{fig:vel05}). We can observe that the backscattered energy 
is concentrated at $\tau=0$, this event correspond to the contribution of $R(\bf{x})^{nb} + R(\bf{x})^{bn}$ as shown in figures ~\ref{fig:cit05_nb}
and \ref{fig:cit05_bn}. We can see that the $R(\bf{x})^{bb}$ does not produces low frequency energy in the image, since it behaves as $R(\bf{x})^{nn}$, but
with opposite slope as shown in figure ~\ref{fig:cit05_bb} .


\subsection{Backscattered energy: does it carry kinematic information?}

To test the sensibility of the backscattered energy to velocity errors I modify the model by making the hard interface in the velocity model. The idea
 is to simulate a salt-sediment interface. I modify the top layer of the model (sediments) in steps of 5\%, I leave the bottom layer constant. The interface
moves accordantly with the image.  Figures ~\rfn{cit01} to ~\rfn{cit09}  show how the backscatterted energy is sensible to the velocity information. One other
aspect to analyze is that the backscattered energy is easy to track since it has vertical slope. The error picking location could be defined as the intersection between
$R(\bf{x},\tau)^{nn}$ with $R(\bf{x},\tau)^{bn} +R(\bf{x},\tau)^{nb}$.




\multiplot{4}{img05_nn,img05_nb,img05_bn,img05_bb}{width=0.4\textwidth}{Understanding the backscattered energy by 
splitting $R(\bf{x})$ shown in figure ~\ref{fig:img05_ref} in: $R(\bf{x})^{nn}$ (a), $R(\bf{x})^{nb}$ (b), $R(\bf{x})^{bn}$ (c) and
$R(\bf{x})^{bb}$ (d).}

\multiplot{4}{cit05_nn,cit05_nb,cit05_bn,cit05_bb}{width=0.2\textwidth}{Understanding the backscattered energy by 
splitting $R(\bf{x},\tau)$ shown in figure ~\ref{fig:cit05_ref} in: $R(\bf{x},\tau)^{nn}$ (a), $R(\bf{x},\tau)^{nb}$ (b), $R(\bf{x},\tau)^{bn}$ (c) and
$R(\bf{x},\tau)^{bb}$ (d).}

\inputdir{flat}

\multiplot*{9}{cit01,cit02,cit03,cit04,cit05,cit06,cit07,cit08,cit09}{width=0.3\textwidth}{Time-lag gathers resulting from introducing velocity errors in the top 
layer of the model: (a) -20\%, (b) -15\%, (c) -10\%, (d) -5\%, (e) 0\%, (f) 5\%, (g) 10\%, (h) 15\% and (i) 20\%. Note how the backscattered energy moves as result
of the velocity error.}











