\section{Introduction}

A seismic image is built under the underlying assumption of single 
scattering,  ~\cite{claerbout:467} introduced the conventional imaging condition which honors this suposition.
 Many methods using this principle has been created, modified and approximated since then. 
The complexity, accuracy and cost of these methods increased over time, walking closely with computational
advances.  

In the early 80's reverse time migration (RTM) was introduced by ~\cite{baysal:1514}, ~\cite{whitmore:382} and ~\cite{GPR:GPR413}.
It wasn't until the late 90's and past decade that we had the computational power to use it in production. RTM produces great images where
other methods failed, this is because of its two way nature which allows it to image infinite dips. The imaging accuracy came with
a new type of undesired events: backscattering noise. People focused their attention to attenuate or eliminate these events.
 Most of the efforts to attenuate these artifacts were performed in the image domain (e.g. ~\cite{guitton:s19}). Others efforts
  were done by modifying the modeling algorithms by introducing
a damping factor in the hard interfaces, thus attenuating the backscattered energy in the wavefields as shown by ~\cite{fletcher:2049}. There 
are also methods for attenuating this events in the extended image domain, via velocity filtering (slope constrains) introduced by ~\cite{kaelin:3125}.

%Others approximations to the imaging condition has been made, making a careful selection of the energy to cross-correlate   as proposed by ~\cite{fei:3130}.

In this paper I focus my attention in understanding the backscattered energy rather than attenuate it or filter it. I go trough some simple
modeling exercises to demonstrate and analyze the relation between these events with the correct velocity model.
