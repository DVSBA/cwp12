\section{Examples}

In this section we illustrate the backscattered events visible on extended images constructed from the Sigsbee 2A model
and clarified based on the criteria introduced in preceding sections. Figure~\ref{fig:Image} shows an image obtained from  a modified
version of the Sigsbee 2A (a salt flood model). We modify the model to avoid backscattering from the base 
of salt. For this example we fix the receiver array on the surface, and we use 100 shots evenly sampled in the surface to build 
the image. For the migration model we use the stratigraphic
velocity which shows sharp interfaces in the sediment section in addition the interface corresponding with the top of salt.

In figure~\ref{fig:Cig6} we see a time-lag gather calculated at $x=19.05km$.
We can see that the gather is very complex; but we can easily identify the backscattered energy pointed with letter ``a" in figure~\ref{fig:Cig6}.
 In this case it maps directly to $\tau=0$ because we use the correct velocity model.
We can also identify the events corresponding to the cross-correlation between reflected waves $\Rct{rr}$ (pointed
with letter ``b"). The $\Rct{rr}$ events have positive slope (given by the sediment velocity at the interface) and are 
visible for $\tau>0$. We can also observe a abrupt change
in the slope of the primary reflection corresponding the sediment-salt interface for the top of salt reflection (shown with
letter ``c").

Figure~\ref{fig:Cigx6} shows a space-lag gather extracted at the same location. The backscattered energy is map toward $\lambda_x=0$ (pointed with 
letter ``a").We see again the $\Rcl{rr}$ case (pointed with letter ``b"), where the energy is mapped away from zero lag.  
Even though we are using the correct model 
we still see energy away from $\lambda_x=0$. This indicates that additional processing is needed before 
we can use space-lag gathers for model update with wave equation tomography.

Figure~\ref{fig:Cip06} shows a common image point extracted at TOS interface at ${\bf c}=(19.05,3.4)km$. Despite of the complexity of this image, we can still
identify similar patterns as shown in figure~\ref{fig:cip05_ref}. The backscattered events are mapped to $\tau>0$ in the
$\tau-\lambda_z$ plane (pointed with letter ``a"). In this plane we can separate with the contribution 
from $\Rce{tr}$ and $\Rce{rt}$, whereas in the common image gathers discussed before we cannot differentiate the individual
contributions (because both cases map to zero lag). The reflection maps as a point to zero lag in the $\tau-\lambda_x$ plane (shown with letter ``c"). 

Understanding the backscattered energy in the extended images for complex scenarios is the first step to use the backscattered events for 
migration velocity analysis. Trough this  report we have used wavefield decomposition to analyze the mapping patterns of the backscattered energy. Although effective, wavefield
decomposition can be very costly, specially in 3D models. In practice we could use filtering of the extended images to isolate the events corresponding to 
the backscattered energy.

\inputdir{flat}
\multiplot{2}{cit05_ref1,citx05_ref1,cip05_ref1,img05_ref}{height=0.2\textheight}{%
Synthetic model example: (a) time-lag gather at x=5km, (b) space-lag gather at x=5km, 
(c) common image point at x=5km, z=1.5km and (d) migrated image of one shot (in x=5km,z=0km) with receivers in the surface} 

%Wavefield movie:
\multiplot{3}{wts-b05-150,wts-b05-275,wts-b05-500,wtr-b05-150,wtr-b05-275,wtr-b05-500,cross_corrs-b_r-b05-150,cross_corrs-b_r-b05-275,cross_corrs-b_r-b05-500,imgs-b_r-b05-150,imgs-b_r-b05-275,imgs-b_r-b05-500}{width=0.3\textwidth}{%
Pictorial explanation of RTM imaging: columns 1, 2 and 3 correspond to three different snapshots at times $t1=0.150s$, $t2=0.275s$ and $t3=0.500s$.
 Rows 1 to 4 correspond to the source wavefield, the receiver wavefield, the multiplication of the source and receiver wavefields,  and the accumulated image over time, respectively.}

%Image splitting:
\multiplot{2}{img05_nn,img05_nb,img05_bn,img05_bb}{height=0.17\textheight}{%
Illustration of the linearity of the conventional imaging condition. We can split the conventional image, figure~\ref{fig:img05_ref} in four %
separate images, $R^{tt}(\xx)$ (a), $R^{tr}(\xx)$ (b), $R^{tr}(\xx)$ (c), and $R^{rr}(\xx)$ (d), corresponding to the correlation of the %
transmitted and/or reflected components of the source and receiver wavefields}

%Time-lag splitting":
\multiplot{2}{cit05_nn,cit05_bn,cit05_nb,cit05_bb}{height=0.28\textheight}{%
Illustration of the linearity of the time-lag extended imaging condition. We can split a time-lag gather, figure~\ref{fig:cit05_ref} in four separate images
, $R^{tt}(z,\tau)$ (a), $R^{rt}(z,\tau)$ (b), $R^{tr}(z,\tau)$ (c) and $R^{rr}(z,\tau)$ (d), corresponding to the correlation of the transmitted %
and/or reflected components of the source and receiver wavefields}

%Space-lag splitting":
\multiplot{2}{citx05_nn,citx05_bn,citx05_nb,citx05_bb}{height=0.28\textheight}{%
Illustration of the linearity of the space-lag extended imaging condition. We can divide figure~\ref{fig:citx05_ref} in four images, %
 $R^{tt}(z,\lambda_x)$ (a), $R^{rt}(z,\lambda_x)$ (b), $R^{tr}(z,\lambda_x)$ (c) and $R^{rr}(z,\lambda_x)$ (d), corresponding to the correlation of %
the transmitted and/or reflected components of the source and receiver wavefields}

%Cip splitting":
\multiplot{2}{cip05_nn,cip05_bn,cip05_nb,cip05_bb}{height=0.2\textheight}{ %
Illustration of the linearity of the extended imaging condition for a common image point. We can decompose a CIP, figure~\ref{fig:cip05_ref}, in four images %
$R^{tt}(\hh,\tau)$ (a), $R^{rt}(\hh,\tau)$ (b), $R^{tr}(\hh,\tau)$ (c), $R^{rr}(\hh,\tau)$ (d) corresponding to the correlation
between transmitted and/or reflected components of the source and receiver wavefields}

\multiplot{3}{cit01_ref,cit02_ref,cit03_ref,cit04_ref,cit05_ref,cit06_ref,cit07_ref,cit08_ref,cit09_ref}{width=0.3\textwidth}{%
Model error sensitivity with time-lag gathers: (a) -12\%, (b) -9\%, (c) -6\%, (d) -3\%, (e) 0\%, (f) 3\%, %
(g) 6\%, (h) 9\% and (i) 12\% velocity perturbation in the top layer. The maximum energy of the backscattered events occur with correct 
velocity shown in panel (e).} 

\multiplot{3}{citx01_ref,citx02_ref,citx03_ref,citx04_ref,citx05_ref,citx06_ref,citx07_ref,citx08_ref,citx09_ref}{width=0.3\textwidth}{%
Model error sensitivity with space-lag gathers: (a) -12\%, (b) -9\%, (c) -6\%, (d) -3\%, (e) 0\%, (f) 3\%, (g) 6\%, (h) 9\% and (i) 12\% %
velocity perturbation in the top layer. Note that the maximum of backscattered energy happens with the correct velocity shown in panel (e).} 

\multiplot{3}{cip01_ref,cip02_ref,cip03_ref,cip04_ref,cip05_ref,cip06_ref,cip07_ref,cip08_ref,cip09_ref}{width=0.3\textwidth}{%
Model error sensitivity with time-lag gathers: (a) -12\%, (b) -9\%, (c) -6\%, (d) -3\%, (e) 0\%, (f) 3\%, %
(g) 6\%, (h) 9\% and (i) 12\% velocity perturbation in the top layer. The backscattered and primary events move away from zero lags,%
 with correct velocity, panel (e), all the events go through zero lags.} 


\multiplot{1}{OF_cit,OF_cix,OF_cip}{height=0.2\textheight}{Objective functions for time-lag exteded image $ J_{\tau}({\bf m})$ (a), space-lag extended image $ J_{\lambda_x}({\bf m})$ (b) and
common image point $J_{\bf c}({\bf m})$.}


\inputdir{sigsbee}
\multiplot{2}{Cig6,Cigx6,Cip06,Image}{height=0.27\textheight}{Sigsbee analysis: time-shift gather (a), space-lag gather (b), common image
point (c) and RTM image (d). The vertical line and thick point shown in the RTM image shows the CIG and CIP locations respectively.}

