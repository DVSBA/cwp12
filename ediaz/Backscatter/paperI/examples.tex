\section{Examples}

In this section we illustrate the backscattered events visible on extended images constructed from the Sigsbee 2A model
and clarified based on the criteria introduced in preceding sections. Figure~\ref{fig:Image} shows an image obtained from  a modified
version of the Sigsbee 2A (a salt flood model). We modify the model to avoid backscattering from the base 
of salt. For this example we fix the receiver array on the surface, and we use 100 shots evenly sampled in the surface to build 
the image. For the migration model we use the stratigraphic
velocity which shows sharp interfaces in the sediment section besides the interface corresponding with the top of salt.

In figure~\ref{fig:Cig6} a 





In figure~\ref{fig:Cig6} we see a common image gather extracted at the vertical line shown in figure ~\ref{fig:Image}.
We can see that the gather is very complex; however, we can easily identify the backscattered energy. In this
case it maps directly to $\tau=0$ because we use the correct velocity model (pointed with letter a in figure~\ref{fig:Cig6}).
We can also identify the events corresponding to the cross-correlation between reflected waves $\Rct{rr}$ (pointed
with letter b). The $\Rct{rr}$ events have positive slope (given by the sediment velocity at the interface) and are 
visible for $\tau>0$. We can also observe a abrupt change
in the slope of the primary reflection corresponding the sediment-salt interface for the top of salt reflection (shown with
letter c in figure~\ref{fig:Cig6}).

Figure~\ref{fig:Cigx6} shows a space-lag gather extracted at the same location. Even though we are using the correct model 
we still see energy away from $\lambda_x=0$. Further processing needs to be performed to use space-lag gathers for model update
when we use a two-way wave equation. We see again the $\Rcl{rr}$ case (pointed in the figure with letter b), 
where the energy is mapped away from zero lag.  We also see a flat event just in the
top of salt (TOS) interface (pointed with letter d), this might be an extended image of a diffraction generated at the TOS interface.

Figure~\ref{fig:Cip06} shows a common image point extracted at TOS interface. Despite of the complexity of this image, we can still
identify similar patterns as shown in figure~\ref{fig:cip05_ref}. The backscattered events are mapped to $\tau>0$ in the
$\tau-\lambda_z$ plane (pointed with letter b in figure~\ref{fig:Cip06}), in this plane we can separate with the contribution 
from $\Rce{tr}$ and $\Rce{rt}$, whereas in the common image gathers discussed before  we cannot differentiate the individual
contributions (because both cases map to zero lag).

\inputdir{flat}
\multiplot{2}{cit05_ref1,citx05_ref1,cip05_ref,img05_ref}{height=0.2\textheight}{Synthetic model example: (a) time-lag gather at x=5km, (b) space-lag gather at x=5km, (c) common image point at x=5km, z=1.5km and (d) migrated image of one shot (in x=5km,z=0km) with receivers in the surface} 

%Wavefield movie:

\multiplot{3}{wts-b05-150,wts-b05-275,wts-b05-500,wtr-b05-150,wtr-b05-275,wtr-b05-500,cross_corrs-b_r-b05-150,cross_corrs-b_r-b05-275,cross_corrs-b_r-b05-500,imgs-b_r-b05-150,imgs-b_r-b05-275,imgs-b_r-b05-500}{width=0.3\textwidth}{RTM imaging exercise: columns 1, 2 and 3 correspond to three different snapshots. Rows 1 to 4 correspond to the source wavefield, the receiver wavefield, the cross-correlation and the accumulated image respectively.}



%Image splitting:
\multiplot{2}{img05_nn,img05_nb,img05_bn,img05_bb}{height=0.17\textheight}{Linearity of the conventional imaging condition. We can split the conventional image ~\rfg{img05_ref} in four cases: $R^{tt}(\xx)$ (a), $R^{tr}(\xx)$ (b), $R^{tr}(\xx)$ (c), $R^{rr}(\xx)$ (d)}

%Time-lag splitting":
\multiplot{2}{cit05_nn,cit05_bn,cit05_nb,cit05_bb}{height=0.28\textheight}{Linearity of the extended imaging condition (time-lags). We split ~\rfg{cit05_ref} in four cases: $R^{tt}(z,\tau)$ (a), $R^{rt}(z,\tau)$ (b), $R^{tr}(z,\tau)$ (c) and $R^{rr}(z,\tau)$ (d)}

%Space-lag splitting":
\multiplot{2}{citx05_nn,citx05_bn,citx05_nb,citx05_bb}{height=0.28\textheight}{Linearity of the extended imaging condition (space-lags). We split \rfg{citx05_ref} in four cases: $R^{tt}(z,\lambda_x)$ (a), $R^{rt}(z,\lambda_x)$ (b), $R^{tr}(z,\lambda_x)$ (c) and $R^{rr}(z,\lambda_x)$ (d)}

%Cip splitting":
\multiplot{2}{cip05_nn,cip05_bn,cip05_nb,cip05_bb}{height=0.2\textheight}{ Split of ~\rfg{cip05_ref} in four cases: $R^{tt}(\lambda_x,\tau)$ (a), $R^{rt}(\lambda_x,\tau)$ (b), $R^{tr}(\lambda_x,\tau)$ (c), $R^{rr}(\lambda_x,\tau)$ (d)}



\multiplot{3}{cit01_ref,cit02_ref,cit03_ref,cit04_ref,cit05_ref,cit06_ref,cit07_ref,cit08_ref,cit09_ref}{width=0.3\textwidth}{Model error sensitivity with time-lag gathers: (a) -12\%, (b) -9\%, (c) -6\%, (d) -3\%, (e) 0\%, (f) 3\%, (g) 6\%, (h) 9\% and (i) 12\% velocity perturbation in the top layer.} 

\multiplot{3}{cip01_ref,cip02_ref,cip03_ref,cip04_ref,cip05_ref,cip06_ref,cip07_ref,cip08_ref,cip09_ref}{width=0.3\textwidth}{Model error sensitivity with time-lag gathers: (a) -12\%, (b) -9\%, (c) -6\%, (d) -3\%, (e) 0\%, (f) 3\%, (g) 6\%, (h) 9\% and (i) 12\% velocity perturbation in the top layer.} 
\multiplot{3}{citx01_ref,citx02_ref,citx03_ref,citx04_ref,citx05_ref,citx06_ref,citx07_ref,citx08_ref,citx09_ref}{width=0.3\textwidth}{Model error sensitivity with space-lag gathers: (a) -12\%, (b) -9\%, (c) -6\%, (d) -3\%, (e) 0\%, (f) 3\%, (g) 6\%, (h) 9\% and (i) 12\% velocity perturbation in the top layer.} 
\multiplot{1}{OF_cit,OF_cix}{height=0.4\textheight}{Objective functions that maximize the backscattered energy at zero lag in the extended images: $ J_{\tau}({\bf s})$ (a) and $ J_{\lambda_x}({\bf s})$ (b).} 



\inputdir{sigsbee}
\multiplot{2}{Image,Cig6,Cigx6,Cip06}{height=0.3\textheight}{Sigsbee analysis: RTM image (a), time-shift gather (b), space-lag gather (c) and common image
point (d). The vertical line and circle shown in the RTM image shows the cig and cip locations respectively.}





