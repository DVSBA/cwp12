\section{Examples}



\inputdir{flat}
\multiplot{3}{cit05_ref,citx05_ref,cip05_ref,img05_ref}{height=0.2\textheight}{Synthetic model example: (a) time-lag gather at x=5km, (b) space-lag gather at x=5km, (c) common image point at x=5km, z=1.5km and (d) migrated image of one shot (in x=5km,z=0km) with receivers in the surface} 




%Wavefield movie:
\multiplot{3}{wts-b05-150,wts-b05-275,wts-b05-500,wtr-b05-150,wtr-b05-275,wtr-b05-500,cross_corrs-b_r-b05-150,cross_corrs-b_r-b05-275,cross_corrs-b_r-b05-500,imgs-b_r-b05-150,imgs-b_r-b05-275,imgs-b_r-b05-500}{width=0.3\textwidth}{RTM imaging exercise: columns 1, 2 and 3 correspond to three different snapshots. Rows 1 to 4 correspond to the source wavefield, the receiver wavefield, the cross-correlation and the accumulated image respectively.}

%Image splitting:
\multiplot{2}{img05_nn,img05_bn,img05_nb,img05_bb}{height=0.17\textheight}{Linearity of the imaging condition. We can split the conventional image (~\rfg{img05_ref}) in four cases: $R^{tt}(\xx)$ (a), $R^{rt}(\xx)$ (b), $R^{tr}(\xx)$ (c), $R^{rr}(\xx)$ (d)}

%Time-lag splitting":
\multiplot{2}{cit05_nn,cit05_bn,cit05_nb,cit05_bb}{height=0.25\textheight}{Linearity of the extended imaging condition (time-lags). We split ~\rfg{cit05_ref} in four cases: $R^{tt}(z,\tau)$ (a), $R^{rt}(z,\tau)$ (b), $R^{tr}(z,\tau)$ (c), $R^{rr}(z,\tau)$ (d)}

%Space-lag splitting":
\multiplot{2}{citx05_nn,citx05_bn,citx05_nb,citx05_bb}{height=0.25\textheight}{Linearity of the extended imaging condition (space-lags). We split \rfg{citx05_ref} in four cases: $R^{tt}(z,\lambda_x)$ (a), $R^{rt}(z,\lambda_x)$ (b), $R^{tr}(z,\lambda_x)$ (c), $R^{rr}(z,\lambda_x)$ (d)}

%Cip splitting":
\multiplot{2}{cip05_nn,cip05_bn,cip05_nb,cip05_bb}{height=0.2\textheight}{ Split of ~\rfg{cip05_ref} in four cases: $R^{tt}(\lambda_x,\tau)$ (a), $R^{rt}(\lambda_x,\tau)$ (b), $R^{tr}(\lambda_x,\tau)$ (c), $R^{rr}(\lambda_x,\tau)$ (d)}






\inputdir{sigsbee}


\multiplot{2}{Image,Cig4,Cigx4}{height=0.3\textheight}{Sigsbee experiment: image shown in (a), green dots show the source locations, a fixed spread was used at the surface. Yellow line in (a) show the location of the time-shift gather in (b) and the space-lag gather in (c).}





