
\inputdir{sigsbee}
\multiplot{4}{Cig6,Cigx6,Cip06,Image}{height=0.23\textheight}{Sigsbee analysis: time-shift gather (a), space-lag gather (b), common image
point (c) and RTM image (d). The vertical line and thick point shown in the RTM image shows the CIG and CIP locations respectively.}


\section{Examples}

In this section we illustrate the backscattered events visible on extended images constructed based on a modified Sigsbee 2A model~\citep{Sigsbee}.
 We modify the model by salt flooding (extending the salt to the bottom
of the model) to avoid backscattering from the base 
of salt, therefore we focus on the reflections from the top of salt only. For this example we fix the receiver array on 
the surface, and we use 100 shots evenly sampled on the surface to build 
the image. For the migration model we use the stratigraphic
velocity which shows sharp interfaces in the sediment section, in addition the interface corresponding with the top of salt.~\rfg{fig:Image}
 shows the conventional image for our modified Sigsbee model, note the strong backscattered
energy above the salt.

Figure~\ref{fig:Cig6} shows a time-lag gather calculated at $x$=$19.05km$.
We can see that the gather is very complex, but we can easily identify the backscattered energy indicated with letter ``a" in Figure~\ref{fig:Cig6}.
 In this case, the backscattered energy maps directly to $\tau=0$ because we use the correct velocity model.
We can also identify the events corresponding to the cross-correlation between reflected waves from the source and receiver side $\Rct{rr}$, indicated
with letter ``b". The $\Rct{rr}$ events have positive slope (given by the sediment velocity at the interface) and are 
visible for $\tau>0$. We can also observe a abrupt change
in the slope of the primary reflection corresponding the sediment-salt interfaces at the top of salt indicated with letter ``c".

Figure~\ref{fig:Cigx6} shows a space-lag common image gather extracted at the same location. The backscattered energy maps toward $\lambda_x=0$, indicated with 
letter ``a". We see again the $\Rcl{rr}$ case, indicated with letter ``b", where the energy is mapped away from zero lag.  
Even though we are using the correct model, we still see energy away from $\lambda_x=0$. This indicates that additional processing is needed before 
we can use space-lag gathers for model update with wave equation tomography.

Figure~\ref{fig:Cip06} shows a common image point extracted at the top of salt interface at $(x,z)=(19.05,3.4)km$. Despite the complexity of this image, we can still
identify similar patterns as shown in Figure~\ref{fig:cip05_ref}. The backscattered events are mapped to $\tau>0$ in the
$\tau-\lambda_z$ plane, indicated with letter ``a". In this plane we can separate with the individual contributions 
from $\Rce{tr}$ (which maps to $\lambda_z<0$ and $\tau>0$), and $\Rce{rt}$ (which maps to $\lambda_z>0$ and $\tau>0$), because they are imaged into two different events, 
whereas in the common image gathers discussed before we cannot differentiate the individual
contributions, because both cases map to zero lag. The image of the reflector maps as a point to zero lag in the $\tau-\lambda_x$ plane (indicated with letter ``c"). 

Understanding the backscattered energy in the extended images for complex scenarios is the first step in using these events for 
migration velocity analysis. In this report, we used wavefield decomposition to analyze the patterns of the backscattered energy in conventional and extended images.
 Although effective, wavefield decomposition can be very costly, specially for 3D models. In practice we need to use filtering of the extended images to isolate the events corresponding to 
the backscattered energy. How we can effectively  implement this operation remains subject to future research.


%%Wavefield movie:
%\multiplot{4}{wts-b05-150,wts-b05-275,wts-b05-500,wtr-b05-150,wtr-b05-275,wtr-b05-500,cross_corrs-b_r-b05-150,cross_corrs-b_r-b05-275,cross_corrs-b_r-b05-500,imgs-b_r-b05-150,imgs-b_r-b05-275,imgs-b_r-b05-500}{angle=90,width=0.16\textwidth}{%
%Pictorial explanation of RTM imaging: columns 1, 2 and 3 correspond to three different snapshots at times $t_1=0.150s$, $t_2=0.275s$ and $t_3=0.500s$.
% Rows 1 to 4 correspond to the source wavefield, the receiver wavefield, the multiplication of the source and receiver wavefields,  and the accumulated image over time, respectively.}
%Wavefield movie:



