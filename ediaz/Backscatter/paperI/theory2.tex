\section{Theory}

\subsection{Conventional imaging condition}
The conventional imaging condition ~\citep{Claerbout:1985:IEI:3887}
is a zero lag cross-correlation between the downgoing source wavefield and the upgoing 
receiver wavefield along the propagation time:
%
\beq
\R=\sum_{shots} \sum_{t} \US(\xx,t)\UR(\xx,t).
\label{eq:cic}
\eeq
%
 Equation~\ref{eq:cic} honors the single scattering or Born assumption. Under this assumption the transmitted
source wavefield generates secondary sources as interacts with the medium discontinuities which propagates in 
space and produce the recorded wavefield at the surface. This assumption means that both the source and receiver
wavefields carry only the transmitted energy.

A wavefield extrapolated with RTM could show, depending on the complexity of the geology, waves traveling in both
upward and downward directions, such as: diving waves, head waves and backscattered waves. The interaction between
 these type of waves contained in the source and receiver wavefields generates new events in the image which are 
commonly referred to as artifacts because they do not follow the geology (i.e earth reflectivity), which is the objective 
of the imaging process. The correlation between forward and backscattered waves is particularly strong when
sharp boundaries are present in the velocity model (e.g salt bodies).

If a sharp boundary is present in the model, we can decompose the source wavefield to transmitted (downgoing) 
 and the reflected (upgoing) energy coming from the sharp boundary:

\beq
\US(\xx,t)= \USr(\xx,t) +\USt(\xx,t),
\label{eq:ssplit}
\eeq
i%
the superscripts ``t" and ``r" stands for transmitted and reflected wavefield respectively. 

The same idea is applied to the receiver wavefield:

\beq
\UR(\xx,t)= \URr(\xx,t) +\URt(\xx,t).
\label{eq:rsplit}
\eeq
%
By taking advantage of the linearity of equation~\ref{eq:cic}  we
can split the conventional imaging condition as follows:

\beq
\R= \Rc{tt}+\Rc{tr}+ \Rc{rt} +\Rc{rr}.
\label{eq:cicsplit}
\eeq
%
By analyzing the individual contributions to the image, we can understand better how the backscattered events
are constructed in the image. This analysis is similar to the one of~\cite{fei:3130}, and~\cite{liu:S29}
whose objective is to filter out the non-geological portions of the image. Here, we approach the problem 
in a broader sense, by attempting to understand the physical meaning of the backscattered energy and its
uses for velocity model building.


\subsubsection{Backscattered events in the conventional image}

In order to gain an understanding of the RTM backscattered events we use a simple model with two-layers and strong velocity
contrasts. In figure~\ref{fig:img05_ref} we show the CIC for one shot at $x=5km$. We can see in this image strong 
backscattered energy above the reflector at $z=1.5km$.

To understand better the origin of the backscattered artifacts we show the imaging steps for our simple model.
In figure 2 we show the RTM imaging process from which we obtain figure~\ref{fig:img05_ref}. Figures~\ref{fig:wts-b05-150},
~\ref{fig:wts-b05-275} and~\ref{fig:wts-b05-500} show three different snapshots
of the source wavefield. Figures~\ref{fig:wtr-b05-150},~\ref{fig:wtr-b05-275} and~\ref{fig:wtr-b05-500}
show the same snapshots for the receiver wavefield. Figures~\ref{fig:cross_corrs-b_r-b05-150},~\ref{fig:cross_corrs-b_r-b05-275} and
~\ref{fig:cross_corrs-b_r-b05-500} show the product between source and receiver wavefields for the same time snapshots. Finally,
figures~\ref{fig:imgs-b_r-b05-150},~\ref{fig:imgs-b_r-b05-275} and~\ref{fig:imgs-b_r-b05-500} show the accumulated image (integration over time of 
the products between wavefields).






We can see in figure~\ref{fig:imgs-b_r-b05-150} the interaction between the transmitted source wavefield $\USt$ in figure~\ref{fig:wts-b05-150} with 
the reflected receiver wavefield $\URr$ in \ref{fig:wtr-b05-150}, in this case the reflected receiver wavefield travels in perfect synchronization
with the transmitted source wavefield, therefore their product shown in figure~\ref{fig:cross_corrs-b_r-b05-150} stacks coherently in the imaging 
process generating the $\Rc{tr}$ contribution to the image $\R$. In the $\Rc{tr}$ image the reflected 
receiver wavefield behaves as the transmitted source wavefield, which is the reason why the backscattered energy is imaged toward the source location. 
%
%
 In the partial image at $t=0.275s$ in figure~\ref{fig:imgs-b_r-b05-0275} we see how we start building the reflector image,
and the source wavefield in figure~\ref{fig:wts-b05-275} reflects, generating new backscattered events
corresponding to the  $\Rc{rt}$ image.
%
%
In the snapshot at $t=0.500s$ the reflector has been imaged already and 
therefore from now on we only add backscattered energy corresponding to the $\Rc{rt}$ image. In this case the 
reflected source wavefield behaves as the receiver wavefield and its energy maps toward the receivers. We can see after
 the imaging process is finished (figure~\ref{fig:img05_ref}) that the backscattered energy is maximum near the critical
 angle range (where the reflected source and receiver wavefields have maximum energy).

By doing wavefield decomposition we can extract exactly the individual contributions of equation~\ref{eq:cicsplit}. In 
figure~\ref{fig:img05_nn} we show the cross-correlation between transmitted wavefields, an image which is due to the
earth reflectivity. In figures~\ref{fig:img05_nb} and~\ref{fig:img05_bn} we see the image of the backscattered
events corresponding to the cases $\Rc{tr}$ and $\Rc{rt}$, again we can observe how the backscattered energy
maps toward the source and the receivers, respectively. The image corresponding to the $\Rc{rr}$ case is shown in 
figure~\ref{fig:img05_bb}, we can see how also contributes to the reflectivity of the earth. \cite{fei:3130} take 
advantage of this analysis to define an image free from backscattered energy as $\R=\Rc{tt}+\Rc{rr}$. Here,
we want to understand better the meaning and uses of the other two images $\Rc{tr}$ and $\Rc{rt}$.


\subsection{Extended imaging condition}

The extended imaging condition ~\citep{sava:S209} is similar to the conventional IC except the cross-correlation
lags between source and receiver wavefield are preserved in the output.
\beq
\Re= \sum_{shots} \sum_{t} \US(\xx - \hh,t-\tau) \UR(\xx+\hh,t+\tau).
\label{eq:eic}
\eeq
%
We can see that the conventional image is a special case of the extended image: $\R=R(\xx,{\bf 0},0)$.

By using extended images, we can measure the accuracy of the velocity model by analyzing the moveout of the events
 \citep{yang:S151}, and we can also perform transformations from the extended domain to angle domain 
\citep{sava:S209,sava:1065,sava:S131}. The extended  images provides the measurement of the similarity between the source
 and receiver wavefields along space and time, so we can exploit these images to analyze and better understand the RTM backscattered events.

In equation~\ref{eq:eic} we observe an increase in the dimensionality of the image, from 3 dimensions to 7 dimensions
if we decide to extend the image in all directions. It is common to perform the analysis 
of extended images at limited locations in order to make this methology feasible for large datasets. 
For cost considerations we often use one extension for common image gathers (CIG) either
in the time-lag axis ($\tau$) or in the space-lag axis ($\lambda_x$). We can also look into common image point gathers (CIP) where we 
fix an observation point ${\bf c}=(x,y,z)$ and analyze the image extensions $\hh,\tau$. 

In presence of sharp velocity interfaces we can follow the same idea of equation~\ref{eq:cicsplit}, and construct 4 separate 
extended images:
\beq
\Re=\Rce{tt}+\Rce{tr}+\Rce{rt}+\Rce{rr}.
\label{eq:eicsplit}
\eeq

In figures ~\ref{fig:cit05_ref1},~\ref{fig:citx05_ref1} and ~\ref{fig:cip05_ref} we show a time-lag gather, a space-lag gather
 and a common image point respectively which represent subsets at fixed surface positions (for CIGs) or fixed space positions 
(for CIPs).


%%%%% ME QUEDE AQUI %%%%%%%%%
\subsubsection{Time-lag gathers}

Using equation~\ref{eq:eicsplit}, we can analyze the individual contributions
for the time-lag gather shown in figure~\ref{fig:cit05_ref1}. Figure~\ref{fig:cit05_nn} shows the image $\Rct{tt}$,
in which we can observe a change in the slope of the events given by the abrupt velocity variation of the model. 
%
Above the reflector depth, the slope is given by the medium velocity of layer 1, and bellow the interface depth the slope is given
 by the velocity of the layer 2. In figures~\ref{fig:cit05_nb} and \ref{fig:cit05_bn} we see the backscattered
 event contributions $\Rct{tr}$ and $\Rct{rt}$ respectively, which indicate that the backscattered event maps towards $\tau=0$
in the extended image. This means that we only get a contribution when we do not dislocate the wavefields by shifting them in time, 
thus reinforcing the idea of wavefield synchronization. We see in the time-lag gathers that the slope of the primaries is very different
from the backscattered slope. ~\cite{kaelin:3125} use the slope difference to filter the backscattered 
events in this domain, and to extract the conventional image from the filtered extended image $\R=R(\xx,\tau=0)$. 
%
%
Figure~\ref{fig:cit05_bb} shows the $\Rct{rr}$ image, in this case the source wavefield is going in upward
direction and the receiver wavefield is going in downward direction, this is similar as if we change the order of cross-correlation in
equation~\ref{eq:eic}. This is why this events map in the time-lag gathers with an slope opposite to the primary above the
interface. The reflected waves only travels in the upper layer, that is why we observe this image above the reflector depth. In the $\Rct{rr}$
image we see to events that map with similar slope, one (the one bellow) has the exact opposite slope as the one shown by the primary reflection,
the other with slightly higher slope (therefore indicates faster velocity) correspond to the interaction between head-waves generated by 
the medium velocity discontinuity.

\subsubsection{Space-lag gathers}

Figure~\ref{fig:citx05_ref1} shows a space-lag gather for the various combinations of the source and receiver wavefield components. 
We note that with the correct velocity model both primaries and backscattered events map to $\lambda_x=0$ since the velocity used for 
imaging is correct. 
%
Figure~\ref{fig:citx05_nn} shows the $\Rcl{tt}$ image, where we see the energy correctly focused at $\lambda_x=0$. 
Figures~\ref{fig:citx05_bn} and ~\ref{fig:citx05_nb} show the backscattered events $\Rcl{bn}$ and $\Rcl{nb}$ in the space-lag gathers, 
which also map toward $\lambda_x=0$.
%
Figure~\ref{fig:citx05_bb} showns the image coming from the reflected wavefields $\Rcl{rr}$, in this case the waves travel only in the first layer.
In addition to the interaction between the geometrical reflections of the source and receiver waves we have to consider as well 
the refracted waves (head-waves in this model). The interaction between head-waves causes the linear events coming out the reflector.  

\subsubsection{Common-image points}

The events involving backscattered energy are also visible in common image point gathers (CIPs).
%
Figure~\ref{fig:cip05_ref} shows a CIP extracted at ${\bf c}=(5,1.5)km$.
Figure~\ref{fig:cip05_nn} shows the CIP for the transmitted wavefields $\Rce{tt}$. We see the energy focusing at zero lag 
for the $\tau-\lambda_x$ panel, the $\lambda_z-\tau$ shows a kink produced by the abrupt change in velocity, not also how the events
shows for negative $\tau$ lags in this panel.  In figure~\ref{fig:cip05_nb}
 shows the $\Rce{tr}$ cip, we can see a change in the $\lambda_z-\tau$ panel, where the backscattered is mapped to positives lags. Figure
~\ref{fig:cip05_bn} shows the complementary backscattered energy, we see that the energy is mapped to negative $\lambda_z$ and positive $\tau$ lags.
The cip from the reflected wavefields shows a weak energy with the same pattern of figure~\ref{fig:cip05_nn}.






\section{Sensitivity of backscattered events to velocity errors}
In this section we test the dependency of the backscattered energy with velocity errors. In previous sections we explained the wavefield 
synchronization idea for correct velocity, this synchronization (mapping toward zero lags) do not hold for incorrect models. 
We introduce a constant error (from -12\% to +12\%) in the velocity of layer 1 to test the sensitivity of the backscattered events.
We move the interface consistently with the velocity used for imaging.
Figures ~\ref{fig:cit01_ref} to \ref{fig:cit09_ref} show time-lag gathers as a function of the velocity error, 
the correct velocity time-lag gather is shown in figure~\ref{fig:cit05_ref}. We can see that the backscattered energy is still mapped vertically, but
away from $\tau=0$. The backscattered events in the time-lag gathers show a kinematic error (i.e the wavefields are not synchronized). We 
see that the backscattered energy move from positive $\tau$ for negative errors to negative $\tau$ values for positive errors. 
%
Figures ~\ref{fig:citx02_ref} to ~\ref{fig:citx09_ref} show a similar display for space-lag gathers, the gather with correct velocity is shown
in figure~\ref{fig:citx05_ref}. 
In this case we can see how both backscattered and primaries energy is maps away from $\lambda_x=0$ when we introduce an error in the model. The backscattered energy maps symetrically
away from zero lag.  

We could use the information contained in the extended images to design an objective function (OF) that uses the presence of backscattered events. 
Minimizing this OF, e.g by wavefield tomography, optimizes the sharp interface positioning (e.g top of salt) and the sediments velocity above the interface.
A straight forward approach based on differential semblance optimization~\citep{shen:2132} uses
 the backscattered energy seen away from zero lags by defining the objective functions for time-lag gathers
\beq
 J_{\tau}({\bf s})= \frac{1}{2} \lnorm{P_{\tau}\left [R^{tr}(\xx,\tau)+R^{rt}(\xx,\tau)\right ]}^2_2,
\label{eq:of1}
\eeq 
and for the space-lag gathers:
\beq
 J_{\lambda_x}({\bf s})= \frac{1}{2} \lnorm{P_{\tau}\left [R^{tr}(\xx,\lambda_x)+R^{rt}(\xx,\lambda)\right ]}^2_2.
\label{eq:of2}
\eeq 
%
Here $P_{\tau}=|\tau|$ and $P_{\lambda_x}=|\lambda_x|$ are functions that penalize the backscattered energy away 
from zero lags, thus defining a residual we need to minimize trough inversion.

The objective functions for our experiment are shown in figures~\ref{fig:OF_cit} and~\ref{fig:OF_citx} for time-lag gathers
and space-lag gathers respectively. One can see that the function minimize at the correct velocity model. In the definition 
of the OF we only use the backscattered energy $\Rct{tr}+\Rct{rt}$, which can be extracted from extended gathers through dip
decomposition. This is a robust and cost effective operation since the various events in the gathers are characterized
by distinct slopes.
