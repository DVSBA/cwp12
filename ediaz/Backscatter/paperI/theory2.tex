\section{Theory}

\subsection{Conventional imaging condition}
The conventional imaging condition ~\citep{Claerbout:1985:IEI:3887}
is a zero lag cross-correlation between the downgoing source wavefield with the upgoing 
receiver wavefield along the propagation time:

\beq
\R=\sum_{shots} \sum_{t} \US(\xx,t)\UR(\xx,t).
\label{eq:cic}
\eeq

Equation ~\ref{eq:cic} honors the single scattering or Born assumption. Under this assumption the transmitted
source wavefield generates secondary sources (as interacts with the medium) which energy is recorded by the 
receiver wavefield at the surface. This means that both wavefields carry only the transmitted energy. 

%In general,
% most of the imaging methodologies (one-way equation migration, Kirchhoff migration, Beam migration, among others)
%follows the Born assumption, this implies that the waves in the source wavefield travels in the downgoing 
%direction and the waves in the receiver wavefield travels in the upgoing direction. In contrast, RTM wavefields
%can handle waves propagating in all directions, this characteristic is due to its extrapolation engine
%(a non factorized wave equation).

A wavefield extrapolated with RTM could show (depending on the complexity of the geology) waves traveling in both
upgoing and downgoing directions such as: diving waves, head waves and backscattered waves. The interaction between
 these type of waves generates new events in the image which are commonly referred as artifacts. The reason for calling
 these events artifacts is because they do not follow the geology (i.e earth reflectivity), which is the objective 
of the imaging process. The interaction between backscattered waves is particularly strong when
sharp boundaries are present in the velocity model (e.g salt bodies).

If a sharp boundary is present in the model, we can decompose the source wavefield in transmitted (downgoing) 
 and the reflected (upgoing) energy coming from the sharp boundary:

\beq
\US(\xx,t)= \USr(\xx,t) +\USt(\xx,t),
\label{eq:ssplit}
\eeq

the superscripts $^t$ and $^r$ stands for transmitted and reflected wavefield respectively. 

The same idea is applied to the receiver wavefield:

\beq
\UR(\xx,t)= \URr(\xx,t) +\URt(\xx,t).
\label{eq:rsplit}
\eeq

By doing this consideration, and by taking advantage of the linearity of equation ~\ref{eq:cic}  we
can split the CIC as follows:

\beq
\R= \Rc{tt}+\Rc{tr}+ \Rc{rt} +\Rc{rr}.
\label{eq:cicsplit}
\eeq

By analyzing the individual contributions to the image we can understand better how the backscattered events
are constructed in the image. A similar analysis is carried by ~\citep{fei:3130,liu:S29} with the objective
of keeping the images that contributes to the geological picture of the earth.

\subsubsection{Backscattered events in the conventional image}

In order to gain an understanding of the RTM imaging process we use a very simple two-layers with a strong velocity
contrast. In figure~\ref{fig:img05_ref} we show the CIC for one shot at $x=5km$. We can see in this image strong 
backscattered energy above the reflector at $z=1.5km$.

In figure 2 we show the RTM imaging process from which we obtain figure~\ref{fig:img05_ref}. Row 1 
(figures ~\ref{fig:wts-b05-150},~\ref{fig:wts-b05-275} and ~\ref{fig:wts-b05-500}) shows three different snapshots
of the source wavefield. Row 2 (figures ~\ref{fig:wtr-b05-150},~\ref{fig:wtr-b05-275} and ~\ref{fig:wtr-b05-500}) 
shows the same snapshots for the receiver wavefield. In the third row we show the product between
source and receiver wavefields. The fourth row shows the accumulated image (integration over time of the panels
in the third row).

We can see in figure~\ref{fig:imgs-b_r-b05-150} the interaction between the transmitted source wavefield $\USt$ (\ref{fig:wts-b05-150}) with 
the reflected receiver wavefield $\URr$ (\ref{fig:wtr-b05-150}), these two wavefields travels in perfect synchronization, therefore its product 
shown in figure~\ref{fig:cross_corrs-b_r-b05-150} stacks coherently in the imaging process generating the $\Rc{tr}$ 
contribution the image $\R$. In the $\Rc{tr}$ case the reflected 
receiver wavefield behaves as the transmitted source wavefield, this is the reason why the backscattered energy 
it is imaged toward the source location. 
In the snapshot at $t=0.275s$ (second column of figure 2) we see how we start building the reflector image,
 and the source wavefield (figure~\ref{fig:wts-b05-275}) reflects and generates new backscattered events
 corresponding to the  $\Rc{rt}$ case. In the snapshot at $t=0.500s$ the reflector has been imaged already and 
therefore we only add after certain time backscattered energy corresponding to the $\Rc{rt}$ image. In this case the 
reflected source wavefield behaves as the receiver wavefield and its energy maps toward the receivers. We can see after
 the imaging process finished (figure~\ref{fig:img05_ref}) that the backscattered energy is maximum near the critical
 angle range (where the reflection coefficient is maximum).

By doing wavefield decomposition we can extract exactly the individual contributions of equation~\ref{eq:cicsplit}. In 
figure~\ref{fig:img05_nn} we show the cross-correlation between transmitted wavefields, this image is the main contribution
to the reflectivity of the earth. In figures~\ref{fig:img05_nb} and~\ref{fig:img05_bn} we see the image of the backscattered
events corresponding to the cases $\Rc{tr}$ and $\Rc{rt}$, again we can observe how the backscattered energy
maps toward the source and the receivers respectively. The image corresponding to the $\Rc{rr}$ case is shown in 
figure~\ref{fig:img05_bb}, we can see how also contributes with the picture of the reflectivity. \cite{fei:3130} takes 
advantage of this analysis to define an image free from backscattered energy as $\R=\Rc{tt}+\Rc{rr}$.

\subsection{Extended imaging condition}

The extended imaging condition ~\citep{sava:S209} is a cross-correlation of the lagged products between source
and receiver wavefields:
\beq
\Re= \sum_{shots} \sum_{t} \US(\xx - \hh,t-\tau) \UR(\xx+\hh,t+\tau),
\label{eq:eic}
\eeq
we can see that the conventional image is a special case of the extended image: $\R=R(\xx,\vec{0},0)$.

By looking extended images we can measure the accuracy of the velocity model by analyzing the moveout of the events
 \citep{yang:S151}, we can also perform transformations from the extended domain to angle domain 
\citep{sava:S209,sava:S131}. The extended  images provides us a measurement of the similarity between the source
 and receiver wavefields along space and time.
 We can exploit these images to analyze and better understand the RTM backscattered events.

In equation~\ref{eq:eic} we observe an increase in the dimensionality of the image, we move from 2 dimensions to 5 dimensions
if we decide to extend the image in all directions. Normally we often use one extension for common image gathers (cig) either
in the time-lag axis $\tau$ or in the space-lag axis $\lambda_x$. We can also look into common image points (cip) where we 
fix an observation point ${\bf c}=(x,z)$ and analyze its extensions $\hh,\tau$. It is common to perform the analysis 
of extended images in limited locations. All these strategies optimize the extended images analysis, making it feasible with 
large datasets.

In presence of sharp velocity interfaces we can follow the same idea of equation~\ref{eq:cicsplit}, and apply it to the EIC:
\beq
\Re=\Rce{tt}+\Rce{tr}+\Rce{rt}+\Rce{rr}.
\label{eq:eicsplit}
\eeq

In figures ~\ref{fig:cit05_ref},~\ref{fig:citx05_ref} and ~\ref{fig:cip05_ref} we show a time-lag gather, a space-lag gather
 and a common image point respectively.

\subsubsection{Time-lag gathers}

We apply the equation~\ref{eq:eicsplit} to analyze the individual contributions
for the time-lag gather shown in figure~\ref{fig:cit05_ref}. In figure~\ref{fig:cit05_nn} we show the contribution between
the transmitted wavefields $\Rct{tt}$, we can observe a change in the slope of the events given by the abrupt velocity variation of the model. 
Above the reflector depth the slope is given by the medium velocity of layer 1, and the slope bellow the interface depth is given
 by the velocity of the layer 2. In figures~\ref{fig:cit05_nb} and \ref{fig:cit05_bn} we see the backscattered
 event contributions $\Rct{tr}$ and $\Rct{rt}$ respectively, we can observe how the backscattered event maps towards $\tau=0$
in the extended image, this means that we only get a contribution when we do not dislocate the wavefields, this re-enforces
the idea of the wavefield synchronization. We see in the time-lag gathers that the slope of the primaries is very different
from the backscattered slope, ~\cite{kaelin:3125} use the slope difference to filter the backscattered 
events in this domain, and then extracting the conventional image from it $\R=R(\xx,\tau=0)$. In figure~\ref{fig:cit05_bb} 
we show the $\Rct{rr}$ contribution, we observe that it maps with a slope similar to the $\Rct{tt}$ case with opposite direction, 
in this case we see the interaction between the geometrical reflections and between the head waves as well. The geometrical reflection
maps with an exact opposite slope as the one observed with the primaries, whereas the head wave maps with a greater slope.

\subsubsection{Space-lag gathers}

In figure~\ref{fig:citx05_ref} we show the space lag gather, note that with the correct velocity model both primaries and
backscattered events map to $\lambda_x=0$. We apply the splitting strategy to decompose the extended image. In figure~\ref{fig:citx05_nn}
we show the $\Rcl{tt}$ case, we see the energy is correctly focused at $\lambda_x=0$. Figures~\ref{fig:citx05_bn} and ~\ref{fig:citx05_nb}
shows the backscattered events in the space-lag gathers, they also map toward $\lambda_x=0$. In figure~\ref{fig:citx05_bb} is shown
the contribution coming from the reflected wavefields $\Rcl{rr}$, in this case the waves travels in above the boundary (in the first layer),
this causes limited focusing due to poor illumination of the event, therefore shows a false apparent velocity error even tough the velocity
is correct. In the space-lag gathers the criteria for correct velocity is shared to both primary and backscattered events.

\subsubsection{Common-image points}

We extract a common image point in the center of the model at ${\bf c}=(5,1.5)km$ shown in figure~\ref{fig:cip05_ref}, this figure 
is an unfolded plot of the cip cube. We split the cip to understand the mapping pattern of the backscattered image. Figure
~\ref{fig:cip05_nn} shows the cip image of the transmitted wavefields $\Rce{tt}$, we see that the energy focus at zero lag 
for the $\tau-\lambda_x$ panel, the $\lambda_z-\tau$ shows a kink produced by the change in velocity. In figure~\ref{fig:cip05_nb}
 shows the $\Rce{tr}$ cip, we can see a change in the $\lambda_z-\tau$ panel, where the backscattered is mapped to positives lags. Figure
~\ref{fig:cip05_bn} shows the complementary backscattered energy, we see that the energy is mapped to negative $\lambda_z$ and positive $\tau$ lags.
The cip from the reflected wavefields shows a weak energy with the same pattern of figure~\ref{fig:cip05_nn}.

\section{Sensitivity of backscattered events to velocity errors}

We introduce a constant error (from -12\% to +12\%) in the layer 1 to test the sensitivity of the backscattered event, in order to generate
reflections in the right position we move the interface accordantly the velocity error. Due to the simplicity of the model the 
depth of the interface also changes linearly. In figures ~\ref{fig:cit02_ref} to \ref{fig:cit09_ref}
 we see the time-lag gathers as a function of the velocity error. We can see that the backscattered energy is still mapped vertically, but
away from $\tau=0$. Figures ~\ref{fig:citx02_ref} to ~\ref{fig:citx09_ref} shows the space-lag gathers as  function of the velocity error, 
in this case we can see how both backscattered and primaries energy is maps away from $\lambda_x=0$. 

We could design a objective function that maximizes the presence of the backscattered events in the image (or minimizes the energy away
zero lag), therefore its model update would optimize the sharp interface positioning (e.g. top of salt) and the velocity above the interface
 (sediment velocity). An straight forward approach based on differential semblance optimization ~\citep{shen:2132} is found by minimizing
 the backscattered energy away from zero lags, we can define an objective function in terms of the slowness model $\bf s$ for time-lag gathers:
\beq
 J_{\tau}({\bf s})= \frac{1}{2} \lnorm{P_{\tau}\left [R^{tr}(\xx,\tau)+R^{rt}(\xx,\tau)\right ]}^2_2,
\label{eq:of1}
\eeq 
or for the space-lag gathers:
\beq
 J_{\lambda_x}({\bf s})= \frac{1}{2} \lnorm{P_{\tau}\left [R^{tr}(\xx,\lambda_x)+R^{rt}(\xx,\lambda)\right ]}^2_2,
\label{eq:of2}
\eeq 

where $P_{\tau}=|\tau|$ and $P_{\lambda_x}=|\lambda_x|$ are functions that penalizes the backscattered energy away 
from zero lags.

The objective function for our simple experiment is shown in figure~\ref{fig:OF_cit} for the time-lag gathers
 and in figure~\ref{fig:OF_cix} for the space-lag gathers. One can see that the objective functions are minimum at the correct velocity model.
As seen in equations~\ref{eq:of1} and ~\ref{eq:of2} we only use the backscattered energy $\Rct{tr}+\Rct{rt}$. 


