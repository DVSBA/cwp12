\section{Introduction}

Reverse time migration is not a new imaging technique ~\citep{baysal:1514, whitmore:382, GPR:GPR413}.
However, it wasn't until the late 90's, and mainly the 00's that computational advances
allowed the geophysical community to use this technology in practice. In general, and 
especially in complex geological settings,  RTM produces better images than other 
imaging methods. Imaging methods (as Kirchhoff migration, one-way equation migration, among others)
are based on approximated solutions to the wave equation.
Kirchhoff migration, a high frequency asymptotic solution to the wave equation, becomes
 unstable with complex velocity models, also fails to handle in an easy way multipathing
trajectories since creates the image based on a single criteria to build the travel time 
(most energetic, or first arrival). Other methods
based on approximations to the wave equation as phase shift migration ~\citep{gazdag:1342} 
rely on a v(z) earth, further approximations are needed to account for lateral variations~\citep{gazdag:124}.
 Besides earth model considerations, one-way wave equation migration propagate wavefields in 
either upward or downward direction, this approximation is inexact as the waves are close
to propagate horizontally, therefore fails to handle properly overturning
waves and reflections from steep dips structures. RTM's propagation engine, the wave equation, 
makes this imaging method very robust and rigorous because it honors the kinematics of 
the wave phenomena by allowing waves to propagate in all directions without any supposition
on the velocity model or the direction of propagation, therefore it takes into account, in a natural way,  
multipathing and  reflections from steep dips.

A striking characteristic of RTM images is the presence of low 
wavenumber events in the image that are not correlated with the geology. The two-way
wave equation simulates scattered waves in all directions, therefore in the imaging condition step
we can see new events corresponding to the cross-correlation between diving waves, head waves
and backscattered waves. The cross-correlation between the backscattered waves is more visible
in presence of sharp boundaries (e.g. top of salt) producing strong events that mask
the image of the earth reflectivity. The backscattered events are considered as noise and have to be
filtered in order to get the earth image.

The seismic industry has dedicated considerable attention to the RTM backscattering events by developing
algorithms and strategies to filter them out of the image. We can classify the filtering approaches in 
two general families: pre-imaging condition and post-imaging condition. The pre-imaging condition
family modify the wavefields (either by modeling or wavefield decomposition) in such a way that
the backscattered events do not form during the imaging process. In the post-imaging family
the artifacts are attenuated by 2D filtering (Laplacian operator, slope filtering, least squares
filtering).

In the pre imaging condition family we can find wavefield decomposition approaches
~\citep[]{liu:S29,fei:3130}. The purpose of wavefield decomposition is to honor the original
assumptions of the imaging condition by cross-correlating downgoing energy with upgoing energy.
We can also find modeling approaches to attenuate reflections in sharp velocity boundaries 
~\citep{fletcher:2049}, or eliminate reflected energy in the modeled wavefields using 
impedance matching ~\citep{baysal:132}. This last method only works with zero-offset (stacked) data.  

The other family of noise attenuation techniques is applied after imaging, this is faster because
it is done after stacking over shots. A common approach is to apply a Laplacian filter
to the image ~\citep{youn:246}. A second strategy proposed by ~\cite{guitton:S19} is a signal-noise
separation by least squares, where the signal is defined as the reflectivity and the noise is the low 
wavenumber energy of the image. 

The low wavenumber events have a special way of mapping into extended images. ~\cite{kaelin:3125} propose
a filter that takes advantage of this feature. The filter is applied to the time-lag gathers, where the
backscattered energy maps into vertical events (discussed later), and therefore is easy to remove.

In this report we analyze the information carried by the backscattered energy. We examine as well 
how it maps into the extended images. We show that by analyzing this piece of information we 
can perform a quality control of sharp boundary placement in the model (e.g. sediment-salt interface). 
We use a very simple 2 layers model to explain how the backscattered events are built in the imaging
process. Then, we test these ideas with an example of the Sigsbee model.

% Talk about Reverse Time Migration
 
% Talk about different techniques to attenuate artifacts

% Introduce the paper objective


%~\cite{claerbout:467} 
%~\cite{baysal:1514}
%~\cite{whitmore:382} 
%~\cite{guitton:S19} 
%~\cite{kaelin:3125} 
%~\cite{fletcher:2049} 
%~\cite{sava:S209} 
%~\cite{fei:3130} 
%~\cite{GPR:GPR413} 
%~\cite{baysal:132}
