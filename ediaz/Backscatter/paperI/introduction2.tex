\section{Introduction}

Reverse time migration is not a new imaging technique ~\citep{baysal:1514, whitmore:382, GPR:GPR413}.
However, it was not until the late 90’s, and mainly the 00’s that computational
23 advances allowed the geophysical community to use this technology in practice. In general, and 
especially in complex geological settings, RTM produces better images than other imaging methods.
 Imaging methods (as Kirchhoff migration, one-way equation migration, among others) are based
 on approximated solutions to the wave equation. Kirchhoff migration, a high frequency asymptotic
 solution to the wave equation, becomes unstable with complex velocity models, also fails to handle
 in an easy way multipathing trajectories since creates the image based on a single criteria to build
 the travel time (most energetic, or first arrival). Other methods based on approximations to the wave
 equation as phase shift migration ~\citep{gazdag:1342} 
rely on a v(z) earth, further approximations are needed to account for lateral variations~\citep{gazdag:124}.
 Besides earth model considerations, one-way wave equation migration propagate wavefields in 
either upward or downward direction, this approximation is inexact as the waves are close
to propagate horizontally, therefore fails to handle properly overturning
waves and reflections from steep dips structures. RTM's propagation engine, a wave equation, 
makes this imaging method very robust and rigorous because it honors the kinematics of 
the wave phenomena by allowing waves to propagate in all directions without any supposition
on the velocity model or the direction of propagation, therefore it takes into account, in a natural way,  
multipathing and  reflections from steep dips.

A striking characteristic of RTM is the presence of low wavenumber events in the image that
 are not correlated with the geology. The two-way wave equation simulates scattered waves in all 
directions, therefore in the imaging condition step we can see new events (that cannot be observed in 
other imaging methods) that corresponds to the cross-correlation between diving waves, head waves 
and backscattered waves. The cross-correlation between the backscattered waves is more visible in
presence of sharp boundaries (e.g. top of salt) producing strong events that mask the image of the 
earth reflectivity. The backscattered events are considered as noise and have to be filtered in order
to get the earth image.

The seismic industry has dedicated considerable attention to the RTM backscattered events by 
developing algorithms and strategies to filter them out of the image. We can classify the filtering
 approaches in two general families: pre-imaging condition and post-imaging condition. The preimaging
 condition family modify the wavefields (either by modeling or wavefield decomposition)
 in such a way that the backscattered events do not form during the imaging process. In the post
imaging family the artifacts are attenuated by 2D filtering (Laplacian operator, slope filtering, least
squares filtering), this approach is considerable cheaper because it works on the image space and 
not on the wavefields.

A common strategy in the pre-imaging condition approaches is wavefield decomposition
~\citep[]{liu:S29,fei:3130}.This method decompose the source and receiver wavefields in upgoing
and downgoing directions. In the imaging step are cross-correlated the part of the wavefields that
contributes to the geology, and discarded the cases that produces backscattered events. Other preimaging 
condition approaches are performed by introducing a modification in the wave equation
 which objective is attenuate the reflections coming from the top of salt 
~\citep{fletcher:2049}. A similar method in the post-stack migration approximation is performed by impedance matching in
the sharp interfaces, therefore it considers a non reflecting wave equation ~\citep{baysal:132}. 

The other family of backscattered energy attenuation techniques is applied after the imaging process.
 A straight forward approach is to apply a Laplacian filter to the image~\citep{youn:246}, 
this filter can be seen as a 2D high pass filter (the backscattered events have a strong low wavenumber
component). A second strategy is a signal to noise separation by least squares filtering, in
 this case the signal is defined as the reflectivity and the noise is the backscattered energy
~\citep{guitton:S19}. Extended imaging condition~\citep{sava:S209} provides information
 of the wavefield similarity for different space and/or time lags.~\cite{kaelin:3125} propose 
that take advantage of the way backscattered events maps into the time-lag gathers. The backscattered
 events maps toward zero time-lag (discussed later) with a correct velocity model whereas the primary 
reflections maps within a limited slopes range (constrained by the velocity model), this difference in
 the slopes allows to design 2D filters that keep events slopes in the primaries reflections range.

In this report we analyze the information carried by the backscattered energy in the extended
 images. We show that the backscattered waves provided important information about the synchronization
 between wavefields. Basically, an image with a correct velocity model will show backscattered
 events with maximum energy. We discuss these ideas with a very simple 2 layers models from
 which we perform an analysis on different extensions of the image. By understanding the patterns
in the simple example we are able to identify and explain similar events in a complex model as the
Sigsbee model. Analyzing extended images in salt environment can be helpful to perform a quality
control on sharp interfaces (either top of salt or bottom of salt).


% Talk about Reverse Time Migration
 
% Talk about different techniques to attenuate artifacts

% Introduce the paper objective


%~\cite{claerbout:467} 
%~\cite{baysal:1514}
%~\cite{whitmore:382} 
%~\cite{guitton:S19} 
%~\cite{kaelin:3125} 
%~\cite{fletcher:2049} 
%~\cite{sava:S209} 
%~\cite{fei:3130} 
%~\cite{GPR:GPR413} 
%~\cite{baysal:132}
