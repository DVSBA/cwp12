\begin{abstract}
RTM backscattered events are produced by the cross-correlation between waves reflected from 
sharp interfaces (e.g top of salt). These events along with head waves and diving waves produces
the so-called RTM artifacts. Commonly these events are seen as a drawback of the RTM method.
Many strategies have been developed to filter them out of the conventional image. The filtering
step is necessary since the backscattered events are not correlated with the geology. 
However, these event can be used to analyze kinematic synchronization between wavefields. 
Preliminary tests in synthetic models shows the sensitivity of the backscattered events
to velocity error. An optimum model in salt environment geological
settings exhibit a maximum energy of the backscattered events in the image when the 
velocity is correct, therefore the velocity model and interfaces interpretation should be carried having in mind
this criterion. The analysis of RTM extended images can be used as a quality control tool 
to constrain salt models and sediment velocity such as the backscattered energy is maximized.
\end{abstract}
