\begin{abstract}
Reverse time migration (RTM) backscattered events are produced by the cross-correlation between waves reflected from 
sharp interfaces (e.g top of salt). These events, along with head waves and diving waves, produce
the so-called RTM artifacts, which are visible as low wavenumber energy on migrated images. 
Commonly, these events are seen as a drawback for the RTM method because they obstruct the geologic
structure ,which is the real objective for imaging.
Many strategies have been developed to filter the artifacts out from the conventional image. However,
 these events contain information that can be used to analyze kinematic synchronization between wavefields reconstructed 
in the subsurface from the source and the receivers. Numeric and theoretical analysis 
indicate the sensitivity of the backscattered energy to velocity accuracy: an accurate velocity model
maximizes the backscattered artifacts. The analysis of RTM extended images can be used as a quality control tool 
and as input to velocity analysis to constrain salt models and sediment velocity.
\end{abstract}
