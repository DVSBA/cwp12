\bibliographystyle{seg}
\def\pcscwp{
Center for Wave Phenomena \\ 
Colorado School of Mines \\ 
psava@mines.edu
}

\def\pcscover{
\author[]{Paul Sava}
\institute{\pcscwp}
\date{}
\logo{WSI}
\large
}

\def\WSI{\textbf{WSI}~}

% ------------------------------------------------------------
% colors
\definecolor{darkgreen}{rgb}{0,0.4,0}
\definecolor{LightGray}{rgb}{0.90,0.90,0.90}
\definecolor{DarkGray}{rgb}{0.85,0.85,0.85}

\definecolor{LightGreen} {rgb}{0.792,1.000,0.439}
\definecolor{LightYellow}{rgb}{1.000,0.925,0.545}
\definecolor{LightBlue}  {rgb}{0.690,0.886,1.000}
\definecolor{LightRed}   {rgb}{1.000,0.752,0.796}

\def\red#1{\textcolor{red}{#1}}
\def\blue#1{\textcolor{blue}{#1}}
\def\green#1{\textcolor{green}{#1}}
\def\darkgreen#1{\textcolor{darkgreen}{#1}}
\def\black#1{\textcolor{black}{#1}}
\def\white#1{\textcolor{white}{#1}}
\def\yellow#1{\textcolor{yellow}{#1}}
\def\gray#1{\textcolor{gray}{#1}}
\def\magenta#1{\textcolor{magenta}{#1}}

% ------------------------------------------------------------
% madagascar
\def\mg{\darkgreen{\sc madagascar~}}
\def\mex#1{ \red{ #1 } }
\def\mvbt#1{\small{\blue{\begin{semiverbatim}#1\end{semiverbatim}}}}

% ------------------------------------------------------------
% equations
\def\bea{\begin{eqnarray}}
\def\eea{  \end{eqnarray}}

\def\beq{\begin{equation}}
\def\eeq{  \end{equation}}


%\def\req#1{(\ref{#1})}

\def\lp{\left (}
\def\rp{\right)}

\def\lb{\left [}
\def\rb{\right]}

\def\pbox#1{ \fbox {$ \displaystyle #1 $}}

\def\non{\nonumber \\ \nonumber}

% ------------------------------------------------------------
% REFERENCE (equations and figures)
\def\rEq#1{Equation~\ref{eqn:#1}}
\def\req#1{equation~\ref{eqn:#1}}
\def\rEqs#1{Equations~\ref{eqn:#1}}
\def\reqs#1{equations~\ref{eqn:#1}}
\def\ren#1{\ref{eqn:#1}}

\def\rFg#1{Figure~\ref{fig:#1}}
\def\rfg#1{Figure~\ref{fig:#1}}
\def\rFgs#1{Figures~\ref{fig:#1}}
\def\rfgs#1{Figures~\ref{fig:#1}}
\def\rfn#1{\ref{fig:#1}}

% ------------------------------------------------------------
% field operators

% trace
\def\tr{\texttt{tr}\;}

% divergence
\def\DIV#1{\nabla \cdot {#1}}

% curl
\def\CURL#1{\nabla \times {#1}}

% gradient
\def\GRAD#1{\nabla {#1}}

% Laplacian
\def\LAPL#1{\nabla^2 {#1}}

\def\dellin{
\lb
\begin{matrix}
\done{}{x} \; \done{}{y} \; \done{}{z}
\end{matrix}
\rb
}

\def\delcol{
\lb
\begin{matrix}
\done{}{x} \non
\done{}{y} \non
\done{}{z}
\end{matrix}
\rb
}

\def\aveclin{
\lb
\begin{matrix}
a_x \; a_y \; a_z
\end{matrix}
\rb
}


% ------------------------------------------------------------

% elastic tensor
\def\CC{{\bf C}}

% identity tensor
\def\I{\;{\bf I}}

% particle displacement vector
\def\uu{{\bf u}}

% particle velocity vector
\def\vv{{\bf v}}

% particle acceleration vector
\def\aa{{\bf a}}

% force vector
\def\ff{{\bf f}}

% wavenumber vector
\def\kk{{\bf k}}

% ray parameter vector
\def\pp{{\bf p}}

% distance vector
\def\hh{ {\boldsymbol{\lambda}} }
\def\xx{{\bf x}}
\def\kkx{{\kk_\xx}}
\def\ppx{{\pp_\xx}}

\def\yy{{\bf y}}

% normal vector
\def\nn{{\bf n}}
\def\ns{\nn_s}
\def\nr{\nn_r}

% source vector
\def\ss{{\bf s}}
\def\kks{{\kk_\ss}}
\def\pps{{\pp_\ss}}

% receiver vector
\def\rr{{\bf r}}
\def\kkr{{\kk_\rr}}
\def\ppr{{\pp_\rr}}

% midpoint vector
\def\mm{{\bf m}}
\def\kkm{{\kk_\mm}}
\def\ppm{{\pp_\mm}}

% offset vector
\def\ho{{\bf h}}
\def\kkh{{\kk_\ho}}
\def\pph{{\pp_\ho}}

% space-lag vector

\def\kkl{{\kk_\hh}}
\def\ppl{{\pp_\hh}}

% CIP vector
\def\cc{ {\bf c}}

% time-lag scalar
\def\tt{\tau}
\def\tts{\tt_s}
\def\ttr{\tt_r}

% frequency
\def\ww{\omega}

%
\def\dd{{\bf d}}

\def\bb{{\bf b}}
\def\qq{{\bf q}}

\def\ii{{\bf i}} % unit vector
\def\jj{{\bf j}} % unit vector

\def\lo{{\bf l}}

% ------------------------------------------------------------

\def\Fop#1{\mathcal{F}     \lb #1 \rb}
\def\Fin#1{\mathcal{F}^{-1}\lb #1 \rb}

% ------------------------------------------------------------
% partial derivatives

\def\dtwo#1#2{\frac{\partial^2 #1}{\partial #2^2}}
\def\done#1#2{\frac{\partial   #1}{\partial #2  }}
\def\dthr#1#2{\frac{\partial^3 #1}{\partial #2^3}}
\def\mtwo#1#2#3{ \frac{\partial^2#1}{\partial #2 \partial#3} }

\def\larrow#1{\stackrel{#1}{\longleftarrow}}
\def\rarrow#1{\stackrel{#1}{\longrightarrow}}

% ------------------------------------------------------------
% elasticity 

\def\stress{\underline{\textbf{t}}}
\def\strain{\underline{\textbf{e}}}
\def\stiffness{\underline{\underline{\textbf{c}}}}
\def\compliance{\underline{\underline{\textbf{s}}}}

\def\GEOMlaw{
\strain = \frac{1}{2} 
\lb \GRAD{\uu} + \lp \GRAD{\uu} \rp^T \rb
}

\def\HOOKElaw{
\stress = \lambda \; tr \lp \strain \rp {\bf I} + 2 \mu \strain 
}

\def\CONSTITUTIVElaw{
\stress = \stiffness \;\strain 
}


\def\NEWTONlaw{
\rho \ddot{\uu} = \DIV{\stress}
}

\def\NAVIEReq{
\rho \ddot\uu =
\lp \lambda + 2\mu \rp \GRAD{\lp \DIV{\uu} \rp}
             - \mu     \CURL{   \CURL{\uu}}
}

% ------------------------------------------------------------

% potentials
\def\VP{\boldsymbol{\psi}}
\def\SP{\theta}

% stress tensor
\def\ssten{{\bf \sigma}}

\def\ssmat{
\lp \matrix {
 \sigma_{11} &  \sigma_{12}   &  \sigma_{13} \cr
 \sigma_{12} &  \sigma_{22}   &  \sigma_{23} \cr
 \sigma_{13} &  \sigma_{23}   &  \sigma_{33} \cr
} \rp
}

% strain tensor
\def\eeten{{\bf \epsilon}}

\def\eemat{
\lp \matrix {
 \epsilon_{11} &  \epsilon_{12}   &  \epsilon_{13} \cr
 \epsilon_{12} &  \epsilon_{22}   &  \epsilon_{23} \cr
 \epsilon_{13} &  \epsilon_{23}   &  \epsilon_{33} \cr
} \rp
}


% plane wave kernel
\def\pwker{A e^{i k \lp \nn \cdot \xx - v t \rp}}


% ------------------------------------------------------------
% details for expert audience (math, cartoons)
\def\expert{
\colorbox{red}{\textbf{\LARGE \white{!}}}
}

% ------------------------------------------------------------
% image, data, wavefields

\def\RR{R}

\def\UU{W}
\def\US{{\UU_s}}
\def\USt{{\UU_s^t}}
\def\USr{{\UU_s^r}}
\def\UR{{\UU_r}}
\def\URt{{\UU_r^t}}
\def\URr{{\UU_r^r}}

\def\DD{D}
\def\DS{{\DD_s}}
\def\DR{{\DD_r}}

\def\UUw{\UU}
\def\USw{{\UU_s}}
\def\URw{{\UU_r}}

\def\DDw{\DD}
\def\DSw{{\DD_s}}
\def\DRw{{\DD_r}}

% perturbations

\def\ds{\Delta s}
\def\di{\Delta \RR}
\def\du{\Delta \UU}

\def\dRR{\Delta \RR}
\def\dUU{\Delta \UU}
\def\dUS{\Delta \US}
\def\dUR{\Delta \UR}

\def\dtt{\Delta \tt}
\def\dhh{\Delta \hh}

% ------------------------------------------------------------
% Green's functions

\def\GG{G}

\def\GS{{\GG_s}}
\def\GR{{\GG_r}}

% ------------------------------------------------------------
% elastic data, wavefields

\def\eRR{\textbf{\RR}}

\def\eDS{{\textbf{\DD}_s}}
\def\eDR{{\textbf{\DD}_r}}
\def\eDD{{\textbf{\DD}}}

\def\eUS{{\textbf{\UU}_s}}
\def\eUR{{\textbf{\UU}_r}}
\def\eUU{{\textbf{\UU}}}

% ------------------------------------------------------------
% sliding bar
\def\tline#1{
\put(95,-3){\small \blue{time}}
\put(-4,-1){\small \blue{0}}
\thicklines
\put( 0,0){\color{blue} \vector(1,0){100}}
\put(#1,0){\color{red}  \circle*{2}}
}

% ------------------------------------------------------------
% arrow on figure
\def\myarrow#1#2#3{
\thicklines
\put(#1,#2){\color{green} \vector(-1,-1){5}}
\put(#1,#2){\color{green} \textbf{#3}}
}

\def\bkarrow#1#2#3{
\thicklines
\put(#1,#2){\color{black} \vector(-1,-1){5}}
\put(#1,#2){\color{black} \textbf{#3}}
}


\def\anarrow#1#2#3#4{
\thicklines
\put(#1,#2){\color{#4} \vector(-1,-1){5}}
\put(#1,#2){\color{#4} \textbf{#3}}
}

% ------------------------------------------------------------
% circle on figure
\def\mycircle#1#2#3{
\thicklines
\put(#1,#2){\color{green} \circle{#3}}
}

% ------------------------------------------------------------
% note on figure
\def\mynote#1#2#3{
\put(#1,#2){\color{green} \textbf{#3}}
}

\def\biglabel#1#2#3{
\put(#1,#2){\Huge \textbf{#3}}
}

\def\wlabel#1#2#3{ \white{ \biglabel{#1}{#2}{#3} }}
\def\klabel#1#2#3{ \black{ \biglabel{#1}{#2}{#3} }}
\def\rlabel#1#2#3{ \red{   \biglabel{#1}{#2}{#3} }}
\def\glabel#1#2#3{ \green{ \biglabel{#1}{#2}{#3} }}
\def\blabel#1#2#3{ \blue { \biglabel{#1}{#2}{#3} }}
\def\ylabel#1#2#3{ \yellow{\biglabel{#1}{#2}{#3} }}

% ------------------------------------------------------------
% centering
\def\cen#1{ \begin{center} \textbf{#1} \end{center}}
\def\cit#1{ \begin{center} \textit{#1} \end{center}}

% emphasis (bold+alert)
\def\bld#1{ \textbf{\alert{#1}}}

% huge fonts
\def\big#1{\begin{center} {\LARGE \textbf{#1}} \end{center}}
\def\hug#1{\begin{center} {\Huge  \textbf{#1}} \end{center}}

% ------------------------------------------------------------
% separator
\def\sep{ \vfill \hrule \vfill}
\def\itab{ \hspace{0.5in}}
\def\nsp{\\ \vspace{0.1in}}

% ------------------------------------------------------------
% integrals

\def\tint#1{\!\!\!\int\!\! #1 dt}
\def\xint#1{\!\!\!\int\!\! #1 d\xx}
\def\wint#1{\!\!\!\int\!\! #1 d\ww}
\def\aint#1{\!\!\!\alert{\int}\!\! #1 d\alert{\xx}}

\def\esum#1{\sum\limits_{#1}}
\def\eint#1{\int\limits_{#1}}

% ------------------------------------------------------------
\def\CONJ#1{\overline{#1}}
\def\MOD#1{\left| {#1} \right|}

% ------------------------------------------------------------
% imaging components

\def\IC{\colorbox{yellow}{\textbf{I.C.}}\;}
\def\WR{\colorbox{yellow}{\textbf{W.R.}}\;}
\def\WE{\colorbox{yellow}{\textbf{W.E.}}\;}
\def\SO{\colorbox{yellow}{\textbf{SOURCE}}\;}
\def\WS{\colorbox{yellow}{\textbf{W.S.}}\;}

% ------------------------------------------------------------
% summary/take home message
\def\thm{take home message}

% ------------------------------------------------------------
\def\dx{\Delta x}
\def\dy{\Delta y}
\def\dz{\Delta z}
\def\dt{\Delta t}

\def\dhx{\Delta h_x}
\def\dhy{\Delta h_y}

\def\kz{{k_z}}
\def\kx{{k_x}}
\def\ky{{k_y}}

\def\kmx{k_{m_x}}
\def\kmy{k_{m_y}}
\def\khx{k_{h_x}}
\def\khy{k_{h_y}}

\def\why{ \alert{\widehat{{\khy}}}}
\def\whx{ \alert{\widehat{{\khx}}}}

\def\lx{{\lambda_x}}
\def\ly{{\lambda_y}}
\def\lz{{\lambda_z}}

\def\klx{k_{\lambda_x}}
\def\kly{k_{\lambda_y}}
\def\klz{k_{\lambda_z}}

\def\mx{{m_x}}
\def\my{{m_y}}
\def\mz{{m_z}}
\def\hx{{h_x}}
\def\hy{{h_y}}
\def\hz{{h_z}}

\def\sx{{s_x}}
\def\sy{{s_y}}
\def\rx{{r_x}}
\def\ry{{r_y}}

% ray parameter (absolute value)
\def\modp#1{\left| \pp_{#1} \right|}

% wavenumber
\def\modk#1{\left| \kk_{#1} \right|}

% ------------------------------------------------------------
\def\kzwk{ {\kz^{\kk}}}
\def\kzwx{ {\kz^{\xx}}}
 
\def\PSk#1{e^{\red{#1 i \kzwk \dz}}}
\def\PSx#1{e^{\red{#1 i \kzwx \dz}}}
\def\PS#1{ e^{\red{#1 i k_z   \dz}}}

\def\TT{t}
\def\TS{t_s}
\def\TR{t_r}

\def\oft{\lp t \rp}
\def\ofw{\lp \ww \rp}

\def\ofx{\lp \xx \rp}
\def\ofk{\lp \kk \rp}
\def\ofs{\lp \ss \rp}
\def\ofr{\lp \rr \rp}
\def\ofz{\lp   z \rp}

\def\ofxt{\lp \xx, t  \rp}
\def\ofst{\lp \ss, t  \rp}
\def\ofrt{\lp \rr, t  \rp}

\def\ofxw{\lp \xx, \ww  \rp}
\def\ofsw{\lp \ss, \ww  \rp}
\def\ofrw{\lp \rr, \ww  \rp}

\def\ofxm{\lp \xx,\hh \rp}

\def\ofxmp{\lp \xx+\hh \rp}
\def\ofxmm{\lp \xx-\hh \rp}

\def\ofmm{\lp \mm      \rp}
\def\ofmz{\lp \mm, z   \rp}
\def\ofmw{\lp \mm, \ww \rp}
\def\ofkm{\lp \kkm     \rp}

% ------------------------------------------------------------
% source/receiver data and wavefields

\def\dst{$\DS\ofst$}
\def\drt{$\DR\ofrt$}
\def\ust{$\US\ofxt$}
\def\urt{$\UR\ofxt$}

\def\dsw{$\DS\ofsw$}
\def\drw{$\DR\ofrw$}
\def\usw{$\US\ofxw$}
\def\urw{$\UR\ofxw$}

% ------------------------------------------------------------
\def\Nx{N_x}
\def\Ny{N_y}
\def\Nz{N_z}
\def\Nt{N_t}
\def\Nw{N_{\ww}}
\def\Nm{N_{\mm}}

\def\Nlx{N_{\lambda_x}}
\def\Nly{N_{\lambda_y}}
\def\Nlz{N_{\lambda_z}}
\def\Nlt{N_{\tau}}

\def\wmin{\ww_{min}}
\def\wmax{\ww_{max}}
\def\zmin{z_{min}}
\def\zmax{z_{max}}
\def\tmin{t_{min}}
\def\tmax{t_{max}}
\def\lmin{\hh_{min}}
\def\lmax{\hh_{max}}
\def\xmin{\xx_{min}}
\def\xmax{\xx_{max}}

% ------------------------------------------------------------
% course qualifiers

\def\fun{\hfill \alert{concepts}}
\def\pra{\hfill \alert{applications}}
\def\fro{\hfill \alert{frontiers}}


% ------------------------------------------------------------
% wavefield extrapolation
\def\ws{ {\ww s} }

\def\kows{\lp \frac{\kx}{\ws} \rp}

\def\kmws{\lp \frac{\modk{\mm}}{\ws} \rp}
\def\kzws{\lp \frac{\kz}       {\ws} \rp}

\def\S{\lb\frac{\modk{\mm}}{\ws  }\rb}
\def\C{\lb\frac{\modk{\mm}}{\ws_0}\rb}
\def\K{\lb\frac{\modk{\mm}}{\ww  }\rb}

\def\Cs{\lb\frac{\modk{\mm}^2}{\lp \ws_0 \rp^2}\rb}

\def\SSR#1{  \sqrt{ \lp \ww {#1} \rp^2 - \modk{\mm}^2} }

\def\SQRsum#1{\sum\limits_{n=1}^{\infty} \lp -1 \rp^n
		\displaystyle{\frac{1}{2} \choose n} #1}

\def\TSE#1#2#3#4{\sum\limits_{#4=#3}^{\infty} \lp -1 \rp^#4
		\displaystyle{#2 \choose #4} {#1}^#4}

\def\onefrac#1#2{\frac{#2^2}{a_#1+b_#1 #2^2}}
\def\SQRfrac#1{
	\sum\limits_{n=1}^{\infty}
	\onefrac{n}{#1} }

\def\dkzds { \left. \frac{d {\kz}}  {d s} \right|_{s_b} }
\def\SSX#1#2{\sqrt{ 1 - \lb \frac{\MOD{#2}}{#1} \rb^2} }
\def\SST#1#2{1 + \sum_{j=1}^N c_j \lb \frac{\MOD{#2}}{#1} \rb^{2j} }

% ------------------------------------------------------------
% acknowledgment
\def\ackcwp{\cen{the sponsors of the\\Center for Wave Phenomena\\at\\Colorado School of Mines}}

% ------------------------------------------------------------
% citation in slides
\def\talkcite#1{{\small \sc #1}}

% ------------------------------------------------------------
\def\ise{GPGN302: Introduction to Electromagnetic and Seismic Exploration}
\def\inv{GPGN409: Inversion}

% ------------------------------------------------------------
\def\model{m}
\def\data {d}

\def\Lop{ {\mathbf{L}}}
\def\Sop{ {\mathbf{S}}}
\def\Eop{ {\mathbf{E}}}
\def\Iop{ {\mathbf{I}}}
\def\Aop{ {\mathbf{A}}}
\def\Pop{ {\mathbf{P}}}
\def\Fop{ {\mathbf{F}}}


% ------------------------------------------------------------
\def\mybox#1{
  \begin{center}
    \fcolorbox{black}{yellow}
    {\begin{minipage}{0.8\columnwidth} {#1} \end{minipage}}
  \end{center}
}

\def\hibox#1{
  \begin{center}
    \fcolorbox{black}{LightGreen}
    {\begin{minipage}{0.8\columnwidth} {#1} \end{minipage}}
  \end{center}
}

% ------------------------------------------------------------
% Nota Bene
\def\nbnote#1{
  \vfill
  \begin{center}
    \colorbox{LightGray}
    {\begin{minipage}{\columnwidth} {\textbf{\black{\large N.B.}} #1} \end{minipage}}
  \end{center}
}

\def\highlight#1{
  \begin{center}
    \colorbox{cyan}
    {\begin{minipage}{\columnwidth} {#1} \end{minipage}}
  \end{center}
}

% ------------------------------------------------------------
\def\pcsshaded#1{
  \definecolor{shadecolor}{rgb}{0.8,0.8,0.8}
  \begin{shaded} {#1} \end{shaded}
  \definecolor{shadecolor}{rgb}{1.0,1.0,1.0}
}

% ------------------------------------------------------------
\def\postit#1{
  \begin{center}
    \colorbox{yellow}
    {\begin{minipage}{0.66\columnwidth} {#1} \end{minipage}} 
  \end{center}
}

% ------------------------------------------------------------
\def\graybox#1{
  \begin{center}
    \colorbox{LightGray}
    {\begin{minipage}{1.00\columnwidth} {#1} \end{minipage}}
  \end{center}
}

\def\whitebox#1{
  \begin{center}
    \colorbox{white}
    {\begin{minipage}{1.00\columnwidth} {#1} \end{minipage}}
  \end{center}
}

\def\yellowbox#1{
  \begin{center}
    \colorbox{LightYellow}
    {\begin{minipage}{1.00\columnwidth} {#1} \end{minipage}}
  \end{center}
}

\def\greenbox#1{
  \begin{center}
    \colorbox{LightGreen}
    {\begin{minipage}{1.00\columnwidth} {#1} \end{minipage}}
  \end{center}
}

\def\bluebox#1{
  \begin{center}
    \colorbox{LightBlue}
    {\begin{minipage}{1.00\columnwidth} {#1} \end{minipage}}
  \end{center}
}

\def\redbox#1{
  \begin{center}
    \colorbox{LightRed}
    {\begin{minipage}{1.00\columnwidth} {#1} \end{minipage}}
  \end{center}
}

\def\hyellow#1{ \colorbox{yellow} #1 }
\def\hgreen #1{ \colorbox{green}  #1 }

% ------------------------------------------------------------
% boxes for vectors and matrices

\def\pcsbox#1#2#3#4{
  % #1 = width
  % #2 = height
  % %3 = hmax
  \begin{picture}(#3,#1)
    \linethickness{0.5mm}
    % 
    \multiput(0,#1)(#3, 0){2}{\line(0,-1){#2}}
    \multiput(0,#1)(0,-#2){2}{\line(+1,0){#3}}
    % 
    \put(1,3){#4}
  \end{picture}
}

\def\pcssym#1#2{
  \begin{picture}(1,#1)
    \put(0,3){#2}
  \end{picture}
}

\def\sidebyside#1#2{
  \begin{center}
    \colorbox{LightBlue}{
      \begin{minipage}{1.0\columnwidth} {#1} \end{minipage}
    }
    \colorbox{LightYellow}{
      \begin{minipage}{1.0\columnwidth} {#2} \end{minipage}
    }
  \end{center}
}

\def\sidebysidebyside#1#2#3{
  \begin{center}
    \colorbox{LightBlue}{
      \begin{minipage}{1.0\columnwidth} {#1} \end{minipage}
    }
    \colorbox{LightYellow}{
      \begin{minipage}{1.0\columnwidth} {#2} \end{minipage}
    }
    \colorbox{LightRed}{
      \begin{minipage}{1.0\columnwidth} {#3} \end{minipage}
    }
  \end{center}
}

\pgfdeclareimage[height=0.75in]{mypic}{esteban}
\bibliography{SEG}

\title[]{Understanding the reverse time migration backscattering:
Noise or signal?}
\subtitle{}
\author[]{Esteban  D\'{i}az$^{*}$ and Paul Sava}
\institute{Center for Wave Phenomena \\
Colorado School of Mines 
\vspace{0.0in} \\ \pgfuseimage{mypic}}
\date{}
\logo{}

\def\big#1{\begin{center} \LARGE \textbf{#1} \end{center}}
\def\cen#1{\begin{center}        \textbf{#1} \end{center}}

% ------------------------------------------------------------
\mode<beamer> { \cwpcover }

% ------------------------------------------------------------


\begin{frame}
Last time we saw:
	\begin{itemize}
		\item Construction of backscattered artifacts in the imaging process. 
		\item Identification in the image, and in the extended images.
		\item Noise or signal? 
	\end{itemize}
\end{frame}

\begin{frame}
What are we going to see now:
	\begin{itemize}
		\item Tests with two models.   
		\item Strategies to isolate the backscattered events without wavefield decomposition.
		\item Future research and tests.
	\end{itemize}
\end{frame}

\inputdir{flat}
\begin{frame}\plot{img05_ref}{width=1\textwidth}{}\end{frame}
\begin{frame}\plot{img05_nn}{width=1\textwidth}{}\end{frame}




\begin{frame} \big{Correct velocity movie} \end{frame}
%-----------------------
 \begin{frame} 
 \begin{columns} 
    \column{0.5\textwidth}
      \plot{imgs-b_r-b05-000}{width=1\textwidth}{
        \klabellarge{20}{45}{Partial image}}
      \plot{wts-b05-000}{width=1\textwidth}{
        \klabellarge{20}{-10}{Source wavefield}}
    \column{0.5\textwidth}
      \plot{cross_corrs-b_r-b05-000}{width=1\textwidth}{
        \klabellarge{20}{45}{Wavefield product}}
      \plot{wtr-b05-000}{width=1\textwidth}{
        \klabellarge{20}{-10}{Receiver wavefield}}
 \end{columns}
\end{frame}
%-----------------------
 \begin{frame} 
 \begin{columns} 
    \column{0.5\textwidth}
      \plot{imgs-b_r-b05-050}{width=1\textwidth}{
        \klabellarge{20}{45}{Partial image}}
      \plot{wts-b05-050}{width=1\textwidth}{
        \klabellarge{20}{-10}{Source wavefield}}
    \column{0.5\textwidth}
      \plot{cross_corrs-b_r-b05-050}{width=1\textwidth}{
        \klabellarge{20}{45}{Wavefield product}}
      \plot{wtr-b05-050}{width=1\textwidth}{
        \klabellarge{20}{-10}{Receiver wavefield}}
 \end{columns}
\end{frame}
%-----------------------
 \begin{frame} 
 \begin{columns} 
    \column{0.5\textwidth}
      \plot{imgs-b_r-b05-100}{width=1\textwidth}{
        \klabellarge{20}{45}{Partial image}}
      \plot{wts-b05-100}{width=1\textwidth}{
        \klabellarge{20}{-10}{Source wavefield}}
    \column{0.5\textwidth}
      \plot{cross_corrs-b_r-b05-100}{width=1\textwidth}{
        \klabellarge{20}{45}{Wavefield product}}
      \plot{wtr-b05-100}{width=1\textwidth}{
        \klabellarge{20}{-10}{Receiver wavefield}}
 \end{columns}
\end{frame}
%-----------------------
 \begin{frame} 
 \begin{columns} 
    \column{0.5\textwidth}
      \plot{imgs-b_r-b05-150}{width=1\textwidth}{
        \klabellarge{20}{45}{Partial image}}
      \plot{wts-b05-150}{width=1\textwidth}{
        \klabellarge{20}{-10}{Source wavefield}}
    \column{0.5\textwidth}
      \plot{cross_corrs-b_r-b05-150}{width=1\textwidth}{
        \klabellarge{20}{45}{Wavefield product}}
      \plot{wtr-b05-150}{width=1\textwidth}{
        \klabellarge{20}{-10}{Receiver wavefield}}
  \end{columns}
\end{frame}
%-----------------------
 \begin{frame} 
 \begin{columns} 
    \column{0.5\textwidth}
      \plot{imgs-b_r-b05-200}{width=1\textwidth}{
        \klabellarge{20}{45}{Partial image}}
      \plot{wts-b05-200}{width=1\textwidth}{
        \klabellarge{20}{-10}{Source wavefield}}
    \column{0.5\textwidth}
      \plot{cross_corrs-b_r-b05-200}{width=1\textwidth}{
        \klabellarge{20}{45}{Wavefield product}}
      \plot{wtr-b05-200}{width=1\textwidth}{
        \klabellarge{20}{-10}{Receiver wavefield}}
 \end{columns}
\end{frame}
%-----------------------
 \begin{frame} 
 \begin{columns} 
    \column{0.5\textwidth}
      \plot{imgs-b_r-b05-250}{width=1\textwidth}{
        \klabellarge{20}{45}{Partial image}}
      \plot{wts-b05-250}{width=1\textwidth}{
        \klabellarge{20}{-10}{Source wavefield}}
    \column{0.5\textwidth}
      \plot{cross_corrs-b_r-b05-250}{width=1\textwidth}{
        \klabellarge{20}{45}{Wavefield product}}
      \plot{wtr-b05-250}{width=1\textwidth}{
        \klabellarge{20}{-10}{Receiver wavefield}}
 \end{columns}
\end{frame}
%-----------------------
 \begin{frame} 
 \begin{columns} 
    \column{0.5\textwidth}
      \plot{imgs-b_r-b05-300}{width=1\textwidth}{
        \klabellarge{20}{45}{Partial image}}
      \plot{wts-b05-300}{width=1\textwidth}{
        \klabellarge{20}{-10}{Source wavefield}}
    \column{0.5\textwidth}
      \plot{cross_corrs-b_r-b05-300}{width=1\textwidth}{
        \klabellarge{20}{45}{Wavefield product}}
      \plot{wtr-b05-300}{width=1\textwidth}{
        \klabellarge{20}{-10}{Receiver wavefield}}
 \end{columns}
\end{frame}
%-----------------------
 \begin{frame} 
 \begin{columns} 
    \column{0.5\textwidth}
      \plot{imgs-b_r-b05-350}{width=1\textwidth}{
        \klabellarge{20}{45}{Partial image}}
      \plot{wts-b05-350}{width=1\textwidth}{
        \klabellarge{20}{-10}{Source wavefield}}
    \column{0.5\textwidth}
      \plot{cross_corrs-b_r-b05-350}{width=1\textwidth}{
        \klabellarge{20}{45}{Wavefield product}}
      \plot{wtr-b05-350}{width=1\textwidth}{
        \klabellarge{20}{-10}{Receiver wavefield}}
 \end{columns}
\end{frame}
%-----------------------
 \begin{frame} 
 \begin{columns} 
    \column{0.5\textwidth}
      \plot{imgs-b_r-b05-400}{width=1\textwidth}{
        \klabellarge{20}{45}{Partial image}}
      \plot{wts-b05-400}{width=1\textwidth}{
        \klabellarge{20}{-10}{Source wavefield}}
    \column{0.5\textwidth}
      \plot{cross_corrs-b_r-b05-400}{width=1\textwidth}{
        \klabellarge{20}{45}{Wavefield product}}
      \plot{wtr-b05-400}{width=1\textwidth}{
        \klabellarge{20}{-10}{Receiver wavefield}}
 \end{columns}
\end{frame}
%-----------------------
 \begin{frame} 
 \begin{columns} 
    \column{0.5\textwidth}
      \plot{imgs-b_r-b05-450}{width=1\textwidth}{
        \klabellarge{20}{45}{Partial image}}
      \plot{wts-b05-450}{width=1\textwidth}{
        \klabellarge{20}{-10}{Source wavefield}}
    \column{0.5\textwidth}
      \plot{cross_corrs-b_r-b05-450}{width=1\textwidth}{
        \klabellarge{20}{45}{Wavefield product}}
      \plot{wtr-b05-450}{width=1\textwidth}{
        \klabellarge{20}{-10}{Receiver wavefield}}
 \end{columns}
\end{frame}
%-----------------------
 \begin{frame} 
 \begin{columns} 
    \column{0.5\textwidth}
      \plot{imgs-b_r-b05-500}{width=1\textwidth}{
        \klabellarge{20}{45}{Partial image}}
      \plot{wts-b05-500}{width=1\textwidth}{
        \klabellarge{20}{-10}{Source wavefield}}
    \column{0.5\textwidth}
      \plot{cross_corrs-b_r-b05-500}{width=1\textwidth}{
        \klabellarge{20}{45}{Wavefield product}}
      \plot{wtr-b05-500}{width=1\textwidth}{
        \klabellarge{20}{-10}{Receiver wavefield}}
 \end{columns}
\end{frame}
%-----------------------
 \begin{frame} 
 \begin{columns} 
    \column{0.5\textwidth}
      \plot{imgs-b_r-b05-550}{width=1\textwidth}{
        \klabellarge{20}{45}{Partial image}}
      \plot{wts-b05-550}{width=1\textwidth}{
        \klabellarge{20}{-10}{Source wavefield}}
    \column{0.5\textwidth}
      \plot{cross_corrs-b_r-b05-550}{width=1\textwidth}{
        \klabellarge{20}{45}{Wavefield product}}
      \plot{wtr-b05-550}{width=1\textwidth}{
        \klabellarge{20}{-10}{Receiver wavefield}}
 \end{columns}
\end{frame}
%-----------------------
 \begin{frame} 
 \begin{columns} 
    \column{0.5\textwidth}
      \plot{imgs-b_r-b05-600}{width=1\textwidth}{
        \klabellarge{20}{45}{Partial image}}
      \plot{wts-b05-600}{width=1\textwidth}{
        \klabellarge{20}{-10}{Source wavefield}}
    \column{0.5\textwidth}
      \plot{cross_corrs-b_r-b05-600}{width=1\textwidth}{
        \klabellarge{20}{45}{Wavefield product}}
      \plot{wtr-b05-600}{width=1\textwidth}{
        \klabellarge{20}{-10}{Receiver wavefield}}
 \end{columns}
\end{frame}




\begin{frame} \frametitle{Defining the backscattered energy.}

\red{The only requirement to get it is to have a hard interface in the model}.

If that is the case, then, we can define:
\beq
\US= \USr + \USt,
\eeq

and
\beq
\UR= \URr + \URt.
\eeq
Using the conventional imaging condition  ~\cite{claerbout:467}:
\beq
R(\xx)= \sum_{shots} \sum_t \US(\xx,t) \UR(\xx,t),
\label{eq:IC}
\eeq

Then,

\beq
R(\xx)= R^{tt}(\xx) +R^{tr}(\xx)+R^{rt}(\xx)+ R^{rr}(\xx).
\label{eq:cases}
\eeq

\end{frame}



\begin{frame}
We can generalize it using the extended imaging condition ~\cite{sava:S209}:
\beq
R({\bf c},\hh,\tau) =  \sum_{shots} \sum_t \US({\bf c}-\hh,t-\tau) \UR({\bf c}+\hh,t+\tau),
\eeq

and,

\beq
R({\bf c},\hh,\tau) = R^{tt}({\bf c},\hh,\tau) +R^{tr}({\bf c},\hh,\tau) +R^{rt}({\bf c},\hh,\tau) +R^{rr}({\bf c},\hh,\tau)^{rr} 
\eeq
\end{frame}


\inputdir{flat}
\begin{frame} \frametitle{splitting the wavefield $\US$}

     %$\US^{b}= \US - \US^{n}$               

    \plot{us_b05_ex}{width=0.5\textwidth}{
      \klabellarge{35}{+40}{$\US^{b} = $}  }
 
 \begin{columns} 
    \column{0.3\textwidth}
      \plot{wts-b05_ex}{width=1\textwidth}{
      \klabellarge{30}{-20}{$(\US  $}  
      \klabellarge{100}{-20}{$ -  $}  }
    \column{0.3\textwidth}
      \plot{wts-n05_ex}{width=1\textwidth}{
      \klabellarge{30}{-20}{$\USt ) $}  }
    \column{0.3\textwidth}
      \plot{mask05_ex}{width=1\textwidth}{
      \klabellarge{20}{-20}{$\times MASK$}  }
\end{columns}
 
 \end{frame}

 \begin{frame} 
 \plot{img05_ref}{height=0.45\textheight}{
 \klabellarge{47}{-5}{$R(\bf{x})$} }

 \begin{columns} 
    \column{0.5\textwidth}
      \plot{citx05_ref}{height=0.45\textheight}{
      \klabellarge{-40}{20}{$R(\bf{x},\bf{\lambda})$} }
    \column{0.5\textwidth}
      \plot{cit05_ref}{height=0.45\textheight}{
      \klabellarge{-40}{20}{$R(\bf{x},\bf{\tau})$} }
      
   \end{columns}
\end{frame}

\begin{frame}
 \plot{img05_ref}{height=0.3\textheight}{
 \wlabellarge{75}{3}{$R(\bf{x})$}}
\begin{columns}
    \column{0.5\textwidth}
      \plot{img05_nn}{width=1\textwidth}{
      \wlabellarge{70}{3}{$R(\bf{x})^{nn}$}}
      \plot{img05_bn}{width=1\textwidth}{
      \wlabellarge{70}{3}{$R(\bf{x})^{rt}$}}

    \column{0.5\textwidth}
      \plot{img05_nb}{width=1\textwidth}{
      \wlabellarge{70}{3}{$R(\bf{x})^{tr}$}}

      \plot{img05_bb}{width=1\textwidth}{
      \wlabellarge{70}{3}{$R(\bf{x})^{bb}$}}

\end{columns}
\end{frame}


\begin{frame}
      \plot{citx05_ref}{height=0.45\textheight}{
       \klabellarge{-50}{50}{$R(\bf{x},\bf{\lambda_x})$} }
\begin{columns}
    \column{0.25\textwidth}
      \plot{citx05_nn}{height=0.35\textheight}{
      \klabellarge{10}{105}{$R(\bf{x},\bf{\lambda_x})^{nn}$} } 
    \column{0.25\textwidth}
      \plot{citx05_nb}{height=0.35\textheight}{
      \klabellarge{10}{105}{$R(\bf{x},\bf{\lambda_x})^{tr}$} }
    \column{0.25\textwidth}
      \plot{citx05_bn}{height=0.35\textheight}{
      \klabellarge{10}{105}{$R(\bf{x},\bf{\lambda_x})^{rt}$} }
    \column{0.25\textwidth}
      \plot{citx05_bb}{height=0.35\textheight}{
      \klabellarge{10}{105}{$R(\bf{x},\bf{\lambda_x})^{bb}$} }

\end{columns}
\end{frame}

\begin{frame}
      \plot{cit05_ref}{height=0.45\textheight}{
       \klabellarge{-50}{50}{$R(\bf{x},\bf{\tau})$} }

\begin{columns}
    \column{0.25\textwidth}
      \plot{cit05_nn}{height=0.35\textheight}{
      \klabellarge{10}{105}{$R(\bf{x},\bf{\tau})^{nn}$} }
    \column{0.25\textwidth}
      \plot{cit05_nb}{height=0.35\textheight}{
      \klabellarge{10}{105}{$R(\bf{x},\bf{\tau})^{tr}$} }
    \column{0.25\textwidth}
      \plot{cit05_bn}{height=0.35\textheight}{
      \klabellarge{10}{105}{$R(\bf{x},\bf{\tau})^{rt}$} }

    \column{0.25\textwidth}
      \plot{cit05_bb}{height=0.35\textheight}{
      \klabellarge{10}{105}{$R(\bf{x},\bf{\tau})^{bb}$} }

\end{columns}
\end{frame}

\begin{frame}
      \plot{cip05_ref}{height=0.45\textheight}{
       \klabellarge{-50}{50}{$R(\bf{x},\bf{\tau})$} }

\begin{columns}
    \column{0.25\textwidth}
      \plot{cip05_nn}{height=0.35\textheight}{
      \klabellarge{10}{105}{$R(\bf{x},\bf{\tau})^{nn}$} }
    \column{0.25\textwidth}
      \plot{cip05_nb}{height=0.35\textheight}{
      \klabellarge{10}{105}{$R(\bf{x},\bf{\tau})^{tr}$} }
    \column{0.25\textwidth}
      \plot{cip05_bn}{height=0.35\textheight}{
      \klabellarge{10}{105}{$R(\bf{x},\bf{\tau})^{rt}$} }

    \column{0.25\textwidth}
      \plot{cip05_bb}{height=0.35\textheight}{
      \klabellarge{10}{105}{$R(\bf{x},\bf{\tau})^{bb}$} }

\end{columns}
\end{frame}



\begin{frame} \frametitle{Test experiments}
\begin{itemize}
   \item 11 velocity models: from 85\% to 115\%
   \item Surface receiver array.
   \item Tests with the hard interface in the velocity
\end{itemize}
\end{frame}
\cwpnote{}


\begin{frame} \frametitle{1D velocity profiles} \plot{vel1d}{height=0.8\textheight}{} \end{frame}





 \begin{frame} 
 \plot{img00_ref}{height=0.45\textheight}{}
 \begin{columns} 
    \column{0.4\textwidth}
      \plot{citx00_ref}{height=0.45\textheight}{}
    \column{0.4\textwidth}
      \plot{cit00_ref}{height=0.45\textheight}{}
   \column{0.2\textwidth}
      \plot{cip00_ref}{height=0.45\textheight}{}
   \end{columns}
\end{frame}
%-----------------------
 \begin{frame} 
 \plot{img01_ref}{height=0.45\textheight}{}
 \begin{columns} 
    \column{0.4\textwidth}
      \plot{citx01_ref}{height=0.45\textheight}{}
    \column{0.4\textwidth}
      \plot{cit01_ref}{height=0.45\textheight}{}
   \column{0.2\textwidth}
      \plot{cip01_ref}{height=0.45\textheight}{}
   \end{columns}
\end{frame}
%-----------------------
 \begin{frame} 
 \plot{img02_ref}{height=0.45\textheight}{}
 \begin{columns} 
    \column{0.4\textwidth}
      \plot{citx02_ref}{height=0.45\textheight}{}
    \column{0.4\textwidth}
      \plot{cit02_ref}{height=0.45\textheight}{}
   \column{0.2\textwidth}
      \plot{cip02_ref}{height=0.45\textheight}{}
   \end{columns}
\end{frame}
%-----------------------
 \begin{frame} 
 \plot{img03_ref}{height=0.45\textheight}{}
 \begin{columns} 
    \column{0.4\textwidth}
      \plot{citx03_ref}{height=0.45\textheight}{}
    \column{0.4\textwidth}
      \plot{cit03_ref}{height=0.45\textheight}{}
   \column{0.2\textwidth}
      \plot{cip03_ref}{height=0.45\textheight}{}
   \end{columns}
\end{frame}
%-----------------------
 \begin{frame} 
 \plot{img04_ref}{height=0.45\textheight}{}
 \begin{columns} 
    \column{0.4\textwidth}
      \plot{citx04_ref}{height=0.45\textheight}{}
    \column{0.4\textwidth}
      \plot{cit04_ref}{height=0.45\textheight}{}
   \column{0.2\textwidth}
      \plot{cip04_ref}{height=0.45\textheight}{}
   \end{columns}
\end{frame}
%-----------------------
 \begin{frame} 
 \plot{img05_ref}{height=0.45\textheight}{}
 \begin{columns} 
    \column{0.4\textwidth}
      \plot{citx05_ref}{height=0.45\textheight}{}
    \column{0.4\textwidth}
      \plot{cit05_ref}{height=0.45\textheight}{}
   \column{0.2\textwidth}
      \plot{cip05_ref}{height=0.45\textheight}{}
   \end{columns}
\end{frame}
%-----------------------
 \begin{frame} 
 \plot{img06_ref}{height=0.45\textheight}{}
 \begin{columns} 
    \column{0.4\textwidth}
      \plot{citx06_ref}{height=0.45\textheight}{}
    \column{0.4\textwidth}
      \plot{cit06_ref}{height=0.45\textheight}{}
   \column{0.2\textwidth}
      \plot{cip06_ref}{height=0.45\textheight}{}
   \end{columns}
\end{frame}
%-----------------------
 \begin{frame} 
 \plot{img07_ref}{height=0.45\textheight}{}
 \begin{columns} 
    \column{0.4\textwidth}
      \plot{citx07_ref}{height=0.45\textheight}{}
    \column{0.4\textwidth}
      \plot{cit07_ref}{height=0.45\textheight}{}
   \column{0.2\textwidth}
      \plot{cip07_ref}{height=0.45\textheight}{}
   \end{columns}
\end{frame}
%-----------------------
 \begin{frame} 
 \plot{img08_ref}{height=0.45\textheight}{}
 \begin{columns} 
    \column{0.4\textwidth}
      \plot{citx08_ref}{height=0.45\textheight}{}
    \column{0.4\textwidth}
      \plot{cit08_ref}{height=0.45\textheight}{}
   \column{0.2\textwidth}
      \plot{cip08_ref}{height=0.45\textheight}{}
   \end{columns}
\end{frame}
%-----------------------
 \begin{frame} 
 \plot{img09_ref}{height=0.45\textheight}{}
 \begin{columns} 
    \column{0.4\textwidth}
      \plot{citx09_ref}{height=0.45\textheight}{}
    \column{0.4\textwidth}
      \plot{cit09_ref}{height=0.45\textheight}{}
   \column{0.2\textwidth}
      \plot{cip09_ref}{height=0.45\textheight}{}
   \end{columns}
\end{frame}
%-----------------------
 \begin{frame} 
 \plot{img10_ref}{height=0.45\textheight}{}
 \begin{columns} 
    \column{0.4\textwidth}
      \plot{citx10_ref}{height=0.45\textheight}{}
    \column{0.4\textwidth}
      \plot{cit10_ref}{height=0.45\textheight}{}
   \column{0.2\textwidth}
      \plot{cip10_ref}{height=0.45\textheight}{}
   \end{columns}
\end{frame}

\begin{frame}
We can define an DSO type objective function (OF) with time-lag gathers:
\beq
    J_{\tau}(s)= \frac{1}{2} \lnorm{P_{\tau}[R^{tr}(\xx,\tau)+R^{rt}(\xx,\tau)]}^2_2,
\eeq
or with space-lag gathers:

\beq
    J_{\hh}(s)= \frac{1}{2} \lnorm{P_{\hh}[R^{tr}(\xx,\hh)+R^{rt}(\xx,\hh)]}^2_2,
\eeq

where $P_\hh = |\hh|$, and $P_\tau=|\tau|$.

\end{frame}



\begin{frame}
    \begin{columns}
      \column{0.3\textwidth}
      \plot{Penalty_cit}{height=0.45\textheight}{}
      \column{0.3\textwidth}
      \plot{Penalty_cix}{height=0.45\textheight}{}
      \column{0.3\textwidth}
      \plot{Penalty_cip}{height=0.45\textheight}{}
    \end{columns}
\end{frame}


\begin{frame}\frametitle{Time-lags OF}
\plot{OF_cit}{width=0.7\textwidth}{}
\end{frame}

\begin{frame}\frametitle{Space-lags OF}
\plot{OF_cix}{width=0.7\textwidth}{}
\end{frame}

\begin{frame}\frametitle{Cip OF}
\plot{OF_cip}{width=0.7\textwidth}{}
\end{frame}


\begin{frame}\frametitle{Lateral gradient model}
\begin{itemize}
    \item Two layers model
    \item Top layer has a lateral gradient
    \item A variable scalar error is applied to the top 
    layer. 
    \item The depth of the interface in the center 
    of the model remains the same.
    \item No wavefield decomposition.
\end{itemize}

\end{frame}
%-----------------------

\inputdir{sigsbee}
 \begin{frame} 
 \plot{Cig6}{height=0.45\textheight}{}
 \begin{columns} 
    \column{0.4\textwidth}
      \plot{Cigx6}{height=0.45\textheight}{}
    \column{0.4\textwidth}
      \plot{Cip06}{height=0.45\textheight}{}
   \column{0.2\textwidth}
      \plot{Image}{height=0.45\textheight}{}
   \end{columns}
\end{frame}



\bibliography{SEG}
