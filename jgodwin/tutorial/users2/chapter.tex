\section{Users 2 - Plotting}

Madagascar also has a robust package of tools for plotting the results of your computations in multi-dimensions! These plots are created using a vector plotting library originally developed by Joe Dellinger at SEP called VPLOT. VPLOT provides a method for making plots that are small in size, aesthetically pleasing, and easily compatible with Latex for rapid creation of production quality images.
[edit]

\subsection{VPLOT}
The VPLOT file format (.vpl suffix) is a self-contained binary data format that describes how to draw the plot on the screen using an interpreter. Since VPLOT is not a standard imaging format, VPLOT files must be viewed with third-party interpreters which we call pens. Each pen interfaces VPLOT with a third-party graphing library such as X11, plplot, opengl, and others. This flexibility makes VPLOT files almost as portable as standard image formats such as: jpeg, png, and gif. Unlike rasterized formats, VPLOT files can be scaled to any size without losing image quality. This feature alone makes VPLOT worth the work, because you don't have to regenerate your images each time you want to change their size!

\subsection{Creating plots}
To generate VPLOT files, we need to pass our computed RSF files through vplot filters, that convert RSF data files to VPLOT files. The vplot filters are named by the type of plot that they produce. The full list of available vplot filters (plot types) is:
\begin{tabular}{| l | l | }
    \hline
sfgraph & create line plots, or scatter plots \\
sfgrey & create raster plots or 2D image plots \\
sfgrey3 & create 3D image plots of panels (or slices) of a 3D cube \\
sfbox & make box-line plots \\
sfcontour & make contour plots \\
sfcontour3 & make contour plots of 3D surfaces \\
sfplotrays & make plots of rays \\
\hline
\end{tabular}
More details about what each plot does, its available parameters and what data inputs each file takes can be found in the appendix.

To actually create a plot, we can use the plotting programs on the command line in the same fashion that we would use a Madagascar program:

\begin{verbatim}
sfspike n1=100 | sfnoise > junk.rsf
sfgraph < junk.rsf title=``noise'' > junk.vpl
\end{verbatim}

\subsection{Visualizing plots}

    In this example, we create a single trace full of noise and then send it to sfgraph to produce a single VPLOT file, junk.vpl.  As you may have noticed, this only creates the file, it does not allow us to actually view the plot, which is useful for saving the plot, but not for visualizing the data.  To visualize the data we need to use a \textbf{pen}, which tells your machine how to actually draw the plot, to view the plot by feeding the VPLOT file to the Madagascar pen. You may have multiple pens installed on your machine, but the only one you need to use is: \textbf{sfpen} which picks the default pen for you.

Use the pen in the following manner:
\begin{verbatim}
sfpen < junk.vpl
\end{verbatim}
This will pop up a screen on your window with the plot shown.  Depending on which pen you are using you may be able to interact with the pen interface to control various parameters of the plot as shown by the buttons at the top of the screen.  As well, there are keyboard shortcuts to many of those commands that are available.  NOTE: oglpen uses a mouse interface that can be accessed by right-clicking on the plot.

\subsection{Converting VPLOT to other formats}

If you want to build reports or documents using other programs, or want to send your images to someone who does not have Madagascar you will need to convert your VPLOT files to other image formats for transfer.  Fortunately, Madagascar includes programs to handle these conversions for you automatically.  To convert a VPLOT plot to another format use the tool \textbf{vpconvert}.  

\textbf{vpconvert} allows you to convert VPLOT files to any of the following formats, provided that you have the appropriate third-party libraries installed: avi eps gif jpeg jpg mpeg mpg pdf png ppm ps svg tif tiff vpl.  Here's an example of how to use vpconvert:
\begin{verbatim}
vpconvert junk.vpl format=jpeg
\end{verbatim}

NOTE: details on how to install these third-party libraries are not included with the Madagascar library, and we provide no support on installing them.  Most users will be able to install them using either package management software (on Linux and Mac) or pre-compiled binaries.

\subsection{Scripting plot generation}

Since plot generation using VPLOT is handled on the command line, we could in theory script plot generation.  However, Madagascar has better utilities built-in to handle automated plot generation as we will discuss in the next few tutorials.
