\section{Introduction to Madagascar}
Welcome to Madagascar.  Madagascar is an open-source software package designed to allow users to quickly build advanced data processing workflows easily and quickly.  However, Madagascar is not trivial to learn, which is why these tutorials exist. Over the course of the next few tutorials we will demonstrate how to use Madagascar, and hopefully show you how powerful Madagascar is as well.

To begin, let's talk about the core principles of Madagascar, and the RSF file format.  

\setlength{\unitlength}{1cm}
\begin{comment}
\begin{figure}
    \begin{picture}(8,8)(0,0)
        \put(0,7){\framebox(8,1){Madagascar}}
        \put(2,6){\framebox(4,1){RSF file format}}
        \put(0,5){\framebox(3,1){something}}
    \end{picture}
\end{figure}
\end{comment}

\subsection{Madagascar's design}

There are a few layers to Madagascar.  At the bottom-most layer, is the RSF file format, which is a common exchange format that all Madagascar programs use.  Non-Madagascar programs can also read/write to and from RSF because it is an open exchange format.  The next level of Madagascar contains the actual Madagascar programs that read and output RSF files and perform data processing.  Concurrent to this level is the VPLOT graphics library which allows users to plot and visualize data in Madagascar.  The scripting utilities in Python and SCons are up another level from the core programs.  These scripting utilities allow users to make powerful scripts that can perform even the most advanced data processing tasks.  The last level of Madagascar is the support which allows users to create reproducible documents using both Madagascar and .  Throughout the course of these tutorials, we will examine all of these components, and demonstrate how they can be used individually, as well as together.  When combined, the individual components of Madagascar allow us to: conduct experiments, process data, make reproducible scripts that we can share with others, and write papers to document our experiments.  Thus, Madagascar is one of the first integrated research environments, that has ``batteries included.''

\subsection{RSF file format}

As previously mentioned, the lowest level of Madagascar is the RSF file format, which is the format used to exchange information between Madagascar programs.  Conceptually, the RSF file format is one of the easiest to understand, as RSF files are simply regularly sampled hypercubes of information.  

RSF hypercubes are defined by two files, the header and the binary file.  The header file contains information about the dimensionality of the hypercube as well as the data contained within the hypercube.  Information contained in the header file includes the following: 
\begin{itemize}
    \item number of elements on all axes,
    \item the origin of the axes,
    \item the sampling interval of elements on the axes,
    \item the type of elements in the axes (i.e. float, integer),
    \item the size of the elements (e.g. single or double precision),
    \item and the location of the actual binary file.
\end{itemize}
Since we often want to view this information about files without deciphering it, we store the header file as an ASCII text file in the local directory with the suffix \textbf{.rsf}.  At any time, you can view or edit the contents of the header files using a text editor such as VIM or Emacs.

The binary file is a file stored remotely (i.e. in a separate directory) that contains the actual hypercube data.  Because the hypercube data can be very large ($10s$ of GB or TB) we usually store the binary files in a remote directory with the suffix \textbf{.rsf@}.  The remote directory is specified by the user using the \textbf{DATAPATH} environmental variable.

\begin{figure}
\setlength{\unitlength}{1cm}
\begin{picture}(12,8)(0,0)
    \put(2,6){\framebox(2,2){Header}}
    \put(3,6){\vector(0,-1){2}}
    \put(2,0){\framebox(10,4){Binary}}
\end{picture}
\caption{Cartoon of the RSF file format.  The header file points to the binary file, which can be separate from one another.  The header file, which is text, is small compared to the binary file.}
\label{fig:rsfformat}
\end{figure}


Because the header and binary are separated from one another, it is possible that we can lose either the header or binary for a particular RSF file.  If the header is lost, then we can simply reconstruct the header using previous knowledge of the data and a text editor.  However, if we lose the binary file, then we cannot reconstruct the data regardless of what we do.  Therefore, you should try and avoid losing either the header or binary data.  The best way to avoid data loss is to make your research reproducible so that your results can be replicated later.

Sometimes though we need to store RSF files for archiving or to transfer to other machines.  Fortunately, we can avoid transferring the header and binary separately by using the combined header/binary format for RSF files.  Files can be constructed using the combined header/binary format by specifying additional parameters on the command line, in particular out=stdout, for any Madagascar program.  The output file will then be header/binary combined, which allows you to transfer the file without fear for losing either the header or binary.  Be careful though: header/binary combined files can be very large, and might slow down your local filesystem.  Best practice is to only use combined header/binary files when absolutely necessary.  


However, the hypercubes are stored in header/binary separate format, meaning that we store header information about the hypercube in the local directory and we store the actual binary data somewhere else.  The advantage to doing this, is that we can store the large binary data file on a fast remote filesystem if we want, and we can avoid working in remote directories.  

\subsection{Additional documentation}

For more complete documentation on the RSF file format see the following links:

\href{http://reproducibility.org/wiki/Guide_to_RSF_file_format}{A gentle guide to the RSF file format.}

\href{http://reproducibility.org/wiki/RSF_Comprehensive_Description}{The full guide to the RSF file format.}

